\documentclass[12pt,]{article}
\usepackage{lmodern}
\usepackage{amssymb,amsmath}
\usepackage{ifxetex,ifluatex}
\usepackage{fixltx2e} % provides \textsubscript
\ifnum 0\ifxetex 1\fi\ifluatex 1\fi=0 % if pdftex
  \usepackage[T1]{fontenc}
  \usepackage[utf8]{inputenc}
\else % if luatex or xelatex
  \ifxetex
    \usepackage{mathspec}
  \else
    \usepackage{fontspec}
  \fi
  \defaultfontfeatures{Ligatures=TeX,Scale=MatchLowercase}
\fi
% use upquote if available, for straight quotes in verbatim environments
\IfFileExists{upquote.sty}{\usepackage{upquote}}{}
% use microtype if available
\IfFileExists{microtype.sty}{%
\usepackage[]{microtype}
\UseMicrotypeSet[protrusion]{basicmath} % disable protrusion for tt fonts
}{}
\PassOptionsToPackage{hyphens}{url} % url is loaded by hyperref
\usepackage[unicode=true]{hyperref}
\hypersetup{
            pdfborder={0 0 0},
            breaklinks=true}
\urlstyle{same}  % don't use monospace font for urls
\usepackage[margin=1in]{geometry}
\usepackage{graphicx,grffile}
\makeatletter
\def\maxwidth{\ifdim\Gin@nat@width>\linewidth\linewidth\else\Gin@nat@width\fi}
\def\maxheight{\ifdim\Gin@nat@height>\textheight\textheight\else\Gin@nat@height\fi}
\makeatother
% Scale images if necessary, so that they will not overflow the page
% margins by default, and it is still possible to overwrite the defaults
% using explicit options in \includegraphics[width, height, ...]{}
\setkeys{Gin}{width=\maxwidth,height=\maxheight,keepaspectratio}
\IfFileExists{parskip.sty}{%
\usepackage{parskip}
}{% else
\setlength{\parindent}{0pt}
\setlength{\parskip}{6pt plus 2pt minus 1pt}
}
\setlength{\emergencystretch}{3em}  % prevent overfull lines
\providecommand{\tightlist}{%
  \setlength{\itemsep}{0pt}\setlength{\parskip}{0pt}}
\setcounter{secnumdepth}{0}
% Redefines (sub)paragraphs to behave more like sections
\ifx\paragraph\undefined\else
\let\oldparagraph\paragraph
\renewcommand{\paragraph}[1]{\oldparagraph{#1}\mbox{}}
\fi
\ifx\subparagraph\undefined\else
\let\oldsubparagraph\subparagraph
\renewcommand{\subparagraph}[1]{\oldsubparagraph{#1}\mbox{}}
\fi

% set default figure placement to htbp
\makeatletter
\def\fps@figure{htbp}
\makeatother

\usepackage{setspace}\doublespacing
\usepackage{float}
\usepackage{caption}
\usepackage{booktabs}
\usepackage{longtable}
\usepackage{array}
\usepackage{multirow}
\usepackage{wrapfig}
\usepackage{float}
\usepackage{colortbl}
\usepackage{pdflscape}
\usepackage{tabu}
\usepackage{threeparttable}
\usepackage{threeparttablex}
\usepackage[normalem]{ulem}
\usepackage{makecell}
\usepackage{xcolor}

\author{}
\date{\vspace{-2.5em}}

\begin{document}

\captionsetup[figure]{labelformat=empty}
\captionsetup[table]{labelformat=empty} \renewcommand{\figurename}{}

\section{Supporting information:}\label{supporting-information}

\section{Recipient and donor characteristics govern hierarchical
structure in a heterospecific pollen competition network of co-flowering
plants}\label{recipient-and-donor-characteristics-govern-hierarchical-structure-in-a-heterospecific-pollen-competition-network-of-co-flowering-plants}

\textbf{Authors:} Jose B. Lanuza, Ignasi Bartomeus, Tia Lynn Ashman,
Romina Rader

The following Supporting Information is available for this article:

\textbf{Table S1.} Species names, common names, varieties and sources of
the different seeds.

\textbf{Table S2.} Numerical values of all the traits measured for each
species.

\textbf{Table S3.} Seed set in percentage for hand cross-pollination,
hand self-pollination, natural selfing and apomixis for all species.

\textbf{Table S4.} Species x species matrix with the significance of
effect ``yes'' or ``no'' of the different donors on the seed set of the
different recipient species.

\textbf{Table S5.} Estimates, standard error, t-value and P-value of the
effect of the different 9 donors on each recipient species.

\textbf{Table S6.} Number of seeds produced with 100\% foreign pollen
treatments for the different recipient species.

\textbf{Table S7.} Phylogenetic signal and significance for all the
different traits

\textbf{Table S8.} Procrustes analysis results.

\textbf{Figure S1.} Correlation matrix for all the different traits.

\textbf{Figure S2.} Total amount of pollen found for the different
treatments.

\textbf{Figure S3.} Pollen ratios for the different recipient species.

\textbf{Figure S4.} Pollen ratios for the different recipient species by
family.

\textbf{Figure S5.} Unipartite bidirectional network with asymmetrical
effect

\textbf{Figure S6.} Statistical comparison of pollen ratios by family as
pollen donor and recipient.

\textbf{Figure S7.} Violin plot of the reproductive biology of the
species.

\textbf{Figure S8.} Grouped effect sizes by family for each recipient
species.

\newpage

\section{TABLE S1}\label{table-s1}

\begingroup\fontsize{7}{9}\selectfont

\begin{longtabu} to \linewidth {>{\raggedright}X>{\raggedright}X>{\raggedright}X>{\raggedright}X}
\toprule
\textbf{Species} & \textbf{Common\_names} & \textbf{Variety} & \textbf{Source}\\
\midrule
\endfirsthead
\multicolumn{4}{@{}l}{\textit{(continued)}}\\
\toprule
Species & Common\_names & Variety & Source\\
\midrule
\endhead

\endfoot
\bottomrule
\endlastfoot
\rowcolor{gray!6}  Brassica oleracea & Wild cabbage & Capitata & https://www.mrfothergills.com.au/\\
\addlinespace
Brassica rapa & Pak choi & Chinensis & https://www.mrfothergills.com.au/\\
\addlinespace
\rowcolor{gray!6}  Eruca sativa & Rocket &  & https://www.mrfothergills.com.au/\\
\addlinespace
Sinapis alba & White mustard &  & https://www.mrfothergills.com.au/\\
\addlinespace
\rowcolor{gray!6}  Ipomoea aquatica & Water spinach &  & https://www.theseedcollection.com.au/\\
\addlinespace
Ipomoea purpurea & Morning glory &  & http://www.shaman-australis.com.au\\
\addlinespace
\rowcolor{gray!6}  Capsicum annuum & Capsicum & California Wonder & https://www.edenseeds.com.au\\
\addlinespace
Petunia integrifolia & Petunia &  & https://www.dianeseeds.com/\\
\addlinespace
\rowcolor{gray!6}  Solanum lycopersicum & Tomato & Tommy Toe & https://www.mrfothergills.com.au/\\
\addlinespace
Solanum melongena & Eggplant & Little Fingers & https://www.4seasonsseeds.com.au/\\*
\end{longtabu}

\endgroup{}

\begin{landscape}

TABLE S2
\begingroup\fontsize{6}{8}\selectfont

\resizebox{\linewidth}{!}{
\begin{tabu} to \linewidth {>{\raggedright\arraybackslash}p{2cm}>{\raggedleft}X>{\raggedleft}X>{\raggedleft}X>{\raggedleft}X>{\raggedleft}X>{\raggedleft}X>{\raggedleft}X>{\raggedleft}X>{\raggedleft}X>{\raggedleft}X>{\raggedleft}X>{\raggedleft}X>{\raggedleft}X}
\toprule
\textbf{Species} & \textbf{Pollen size $\mu$m} & \textbf{Pollen grains per anther} & \textbf{Ovule number} & \textbf{Pollen:ovule ratio} & \textbf{Stigma area $\mu$m$^{2}$} & \textbf{Stigma length (mm)} & \textbf{Stigma width (mm)} & \textbf{Style length (mm)} & \textbf{Style width (mm)} & \textbf{Ovary length (mm)} & \textbf{Ovary width (mm)} & \textbf{Selfing rate} & \textbf{SI index}\\
\midrule
\rowcolor{gray!6}  Brassica oleracea & 27.72 & 42033 & 29 & 8696.48 & 0.62 & 0.53 & 0.88 & 2.32 & 0.65 & 5.93 & 1.11 & 0.0 & 0.00\\
\addlinespace
Brassica rapa & 25.35 & 7133 & 26 & 1646.08 & 0.36 & 0.37 & 0.73 & 1.08 & 0.52 & 3.53 & 0.88 & 0.0 & 0.00\\
\addlinespace
\rowcolor{gray!6}  Capsicum annuum & 32.46 & 30761 & 241 & 765.83 & 1.06 & 0.72 & 1.18 & 3.24 & 1.06 & 3.15 & 5.80 & 0.8 & 0.64\\
\addlinespace
Eruca versicaria & 24.95 & 22151 & 24 & 5537.75 & 0.35 & 0.73 & 0.67 & 6.60 & 0.73 & 4.42 & 0.94 & 0.1 & 0.02\\
\addlinespace
\rowcolor{gray!6}  Ipomoea aquatica & 70.10 & 858 & 4 & 1072.50 & 3.26 & 1.43 & 2.25 & 19.44 & 0.45 & 2.38 & 1.42 & 0.6 & 0.75\\
\addlinespace
Ipomoea purpurea & 97.59 & 654 & 6 & 545.00 & 2.27 & 1.24 & 1.88 & 28.23 & 0.58 & 1.06 & 1.57 & 1.0 & 2.74\\
\addlinespace
\rowcolor{gray!6}  Petunia integrifolia & 24.74 & 34657 & 220 & 787.66 & 1.17 & 0.80 & 1.32 & 14.65 & 0.45 & 3.13 & 1.77 & 0.9 & 0.26\\
\addlinespace
Sinapis alba & 33.59 & 3507 & 6 & 3507.00 & 0.55 & 0.63 & 0.91 & 3.62 & 0.77 & 1.98 & 1.07 & 0.7 & 1.12\\
\addlinespace
\rowcolor{gray!6}  Solanum lycopersicum & 22.00 & 28915 & 92 & 1885.76 & 0.09 & 0.19 & 0.35 & 6.47 & 0.31 & 1.16 & 1.13 & 0.7 & 0.48\\
\addlinespace
Solanum melongena & 25.18 & 166989 & 1010 & 992.01 & 1.14 & 0.96 & 1.33 & 11.33 & 0.94 & 4.02 & 3.55 & 1.0 & 1.45\\
\bottomrule
\end{tabu}}
\endgroup{}

\end{landscape}

TABLE S3 \begingroup\fontsize{6}{8}\selectfont

\resizebox{\linewidth}{!}{
\begin{tabu} to \linewidth {>{\raggedright\arraybackslash}p{2cm}>{\raggedleft}X>{\raggedleft}X>{\raggedleft}X>{\raggedleft}X}
\toprule
\textbf{Species} & \textbf{Hand cross-pollination} & \textbf{Hand self-pollination} & \textbf{Spontaneous selfing} & \textbf{Apomixis}\\
\midrule
\rowcolor{gray!6}  Brassica oleracea & 32.07 & 0.00 & 0.00 & 0\\
\addlinespace
Brassica rapa & 44.97 & 0.00 & 0.00 & 0\\
\addlinespace
\rowcolor{gray!6}  Capsicum annuum & 80.00 & 56.47 & 19.34 & 0\\
\addlinespace
Eruca sativa & 23.75 & 0.42 & 0.00 & 0\\
\addlinespace
\rowcolor{gray!6}  Ipomoea aquatica & 40.00 & 30.00 & 20.00 & 0\\
\addlinespace
Ipomoea purpurea & 31.67 & 86.67 & 31.67 & 0\\
\addlinespace
\rowcolor{gray!6}  Petunia integrifolia & 80.16 & 24.77 & 0.00 & 0\\
\addlinespace
Sinapis alba & 41.67 & 48.33 & 5.00 & 15\\
\addlinespace
\rowcolor{gray!6}  Solanum lycopersium & 85.65 & 41.20 & 68.48 & 0\\
\addlinespace
Solanum melongena & 60.48 & 74.87 & 21.56 & 0\\
\bottomrule
\end{tabu}} \endgroup{}

\newpage

\begin{figure}
\centering
\includegraphics{Supp_Material_files/figure-latex/unnamed-chunk-4-1.pdf}
\caption{Fig S1}
\end{figure}

\clearpage

\begin{figure}
\centering
\includegraphics{Supp_Material_files/figure-latex/unnamed-chunk-5-1.pdf}
\caption{\textbf{Fig S3} Average proportion of heterospecific pollen per
family for the different 20 treatments counted. On the label of the
y-axis, the first family is the pollen recipient family, and the second,
the pollen donor family. These pollen ratios are the number of
heterospecific pollen grains divided by the total pollen grains per
stigma (conspecific and heterospecific pollen) and then averaged by
family (N=20). The vertical bar on intercept 50, represents equal
proportions of both recipient (grey) and donor (light blue) pollen.}
\end{figure}

\clearpage

\includegraphics{Supp_Material_files/figure-latex/unnamed-chunk-6-1.pdf}

\clearpage

\begin{figure}
\centering
\includegraphics{Supp_Material_files/figure-latex/unnamed-chunk-7-1.pdf}
\caption{\textbf{Fig. S4} Pollen ratios comparisons between the
different pollen recipient families where the boxes represent least
square means, the error bars, confidence intervals 95\%, and sharing
numbers indicate no significant differences between groups (Tukey
adjusted comparisons). These pollen ratios are the total number of
heterospecific pollen grains divided by the total quantity of pollen
(conspecific pollen + heterospecific pollen), and then compared by
family (N=20).}
\end{figure}

\newpage

\begin{figure}
\centering
\includegraphics{Supp_Material_files/figure-latex/unnamed-chunk-8-1.pdf}
\caption{\textbf{Fig. S5} Unipartite bidirectional network with
asymmetrical effect. The lines with the arrow heads connect the impact
of foreign pollen (effect size) of each pollen donor species on each
recipient species. All the arrow heads point to the recipient species of
the reciprocal interaction. Lines of species that did not have a
negative impact are not represented. The different nodes and the donor
species appear coloured by family: Solanaceae (orange), Brassicaceae
(blue) and Convolvulaceae (green). The intensity of the effect is
represented by the line´s size where a larger effect size corresponds to
a thicker line and a thinner line to a smaller effect size. Species
code: BROL: Brassica oleracea, BRRA: Brassica rapa, ERSA: Eruca sativa,
SIAL: Sinapis alba, IPAQ: Ipomoea aquatica, IPPU: Ipomoea purpurea,
CAAN: Capsicum annuum, PEIN: Petunia integrifolia, SOLY: Solanum
lycopersicum, SOME: Solanum melongena.}
\end{figure}

\clearpage

\begin{figure}
\centering
\includegraphics{Supp_Material_files/figure-latex/unnamed-chunk-9-1.pdf}
\caption{\textbf{Fig S6} Graphical representation of the correlation
matrix of the different reproductive traits considered in the
experiment. Positive correlations are displayed in blue and negative in
red. The intensity, size and colour of the circles are proportional to
the correlation coefficient from Pearson's r.}
\end{figure}

\clearpage

\begin{figure}
\centering
\includegraphics{Supp_Material_files/figure-latex/unnamed-chunk-10-1.pdf}
\caption{\textbf{Fig. S7} Violin plot of the proportion of seeds
coverted to ovule (\%) for all species with four different
hand-pollination treatments (apomixis, hand cross pollination, hand self
pollination and spontaneous selfing). The coloured dots, represent the
different values of seed set for each treatment.}
\end{figure}

\newpage

\begin{figure}
\centering
\includegraphics{Supp_Material_files/figure-latex/unnamed-chunk-11-1.pdf}
\caption{\textbf{Fig. S8} Effect sizes (95\% confidence intervals) of
the different families on each focal species (recipient) and a hand
cross-pollination treatment (control). For each species the control
treatment appears in yellow and the grouped effect per family in blue
for Brassicaceae, green for Convolvulaceae and orange for Solanaceae.}
\end{figure}

\end{document}
