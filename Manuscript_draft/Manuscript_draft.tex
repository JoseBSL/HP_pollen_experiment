%%%%%%%%%%%%%%%%%%%%%%%%%%%%%%%%%%%%%%%%%%%%%%%%%%%%%%%%%%%%%%%%%%%%%%%%%%%%%
%% Original default rstudio/pandoc latex file
%% upated by @jhollist 09/15/2014
%% inspired by @cboetting https://github.com/cboettig/template and
%% @rmflight blog posts:
%% http://rmflight.github.io/posts/2014/07/analyses_as_packages.html 
%% http://rmflight.github.io/posts/2014/07/vignetteAnalysis.html).  
%%%%%%%%%%%%%%%%%%%%%%%%%%%%%%%%%%%%%%%%%%%%%%%%%%%%%%%%%%%%%%%%%%%%%%%%%%%%%

\documentclass[11pt,a4paper]{article}
\usepackage[T1]{fontenc}
\usepackage{lmodern}
\usepackage{amssymb,amsmath}
\usepackage{ifxetex,ifluatex}
\usepackage{fixltx2e} % provides \textsubscript
% use upquote if available, for straight quotes in verbatim environments
\IfFileExists{upquote.sty}{\usepackage{upquote}}{}
\ifnum 0\ifxetex 1\fi\ifluatex 1\fi=0 % if pdftex
  \usepackage[utf8]{inputenc}
\else % if luatex or xelatex
  \ifxetex
    \usepackage{mathspec}
    \usepackage{xltxtra,xunicode}
  \else
    \usepackage{fontspec}
  \fi
  \defaultfontfeatures{Mapping=tex-text,Scale=MatchLowercase}
  \newcommand{\euro}{€}
\fi
% use microtype if available
\IfFileExists{microtype.sty}{\usepackage{microtype}}{}
\usepackage[margin=1in]{geometry}
\ifxetex
  \usepackage[setpagesize=false, % page size defined by xetex
              unicode=false, % unicode breaks when used with xetex
              xetex]{hyperref}
\else
  \usepackage[unicode=true]{hyperref}
\fi
\hypersetup{breaklinks=true,
            bookmarks=true,
            pdfauthor={},
            pdftitle={Working title, The impact of heterospecific pollen on plant reproductive success is mediated by phylogenetic distance and floral reproductive traits},
            colorlinks=true,
            citecolor=blue,
            urlcolor=blue,
            linkcolor=magenta,
            pdfborder={0 0 0}}
\urlstyle{same}  % don't use monospace font for urls
\setlength{\parindent}{0pt}
\setlength{\parskip}{6pt plus 2pt minus 1pt}
\setlength{\emergencystretch}{3em}  % prevent overfull lines
\setcounter{secnumdepth}{0}

%%%%%%%%%%%%%%%%%%%%%%%%%%%%%%%%%%%%%%%%%%%%%%%%%%%%%%%%
%Changes borrowed from @cboettig, added by @jhollist 
% A modified page layout 
\textwidth 6.75in
\oddsidemargin -0.15in
\evensidemargin -0.15in
\textheight 9in
\topmargin -0.5in
\usepackage{lineno} % add 
  \linenumbers % turns line numbering on 
%%%%%%%%%%%%%%%%%%%%%%%%%%%%%%%%%%%%%%%%%%%%%%%%%%%%%%%%

%%%%%%%%%%%%%%%%%%%%%%%%%%%%%%%%%%%%%%%%%%%%%%%%%%%%%%%%
%%Packages and layout changes by @jhollist 09/15/2014
\usepackage{ragged2e}
\usepackage[font=normalsize]{caption}
  \usepackage[doublespacing]{setspace}
\usepackage{parskip}
\usepackage{fancyhdr}
\pagestyle{fancy}
\fancyhf{}
\renewcommand{\headrulewidth}{0pt}
  \rfoot{\today}
\lfoot{\thepage}
%%Changed default abstract width and added lines
\renewenvironment{abstract}{
  \hfill\begin{minipage}{1\textwidth}
  \rule{\textwidth}{1pt}\vspace{5pt}
  \normalsize
  \begin{justify}
  \bfseries\abstractname\vspace{5pt}
  \end{justify}}
  {\par\noindent\rule{\textwidth}{1pt}\end{minipage}
}
%%%%%%%%%%%%%%%%%%%%%%%%%%%%%%%%%%%%%%%%%%%%%%%%%%%%%%%%

\title{Working title, The impact of heterospecific pollen on plant reproductive
success is mediated by phylogenetic distance and floral reproductive
traits}
\author{
Jose B. Lanuza, Ignasi Bartomeus, Tia-Lynn Ashman, Romina Rader
}
\date{}
% Allowing for landscape pages
\usepackage{lscape}
\newcommand{\blandscape}{\begin{landscape}}
\newcommand{\elandscape}{\end{landscape}}

% Left justification of the text: see https://www.sharelatex.com/learn/Text_alignment
% \usepackage[document]{ragged2e} % already in the latex template
\newcommand{\bleft}{\begin{flushleft}}
\newcommand{\eleft}{\end{flushleft}}

%%Fix tightlist error: https://stackoverflow.com/questions/40438037/tightlist-error-using-pandoc-with-markdown
%%Added 2018-03-26 
\providecommand{\tightlist}{%
  \setlength{\itemsep}{0pt}\setlength{\parskip}{0pt}}
%%%  
  

\begin{document}
%%Edited by @jhollist 09/15/2014
%%Adds title from YAML
\begin{singlespace}
\begin{center}
\huge Working title, The impact of heterospecific pollen on plant reproductive
success is mediated by phylogenetic distance and floral reproductive
traits
\end{center}
%%Adds Author, correspond email asterisk, and affilnum from YAML
\begin{center}
\large
Jose B. Lanuza, Ignasi Bartomeus, Tia-Lynn Ashman, Romina Rader \textsuperscript{*} \textsuperscript{1,2,3} 
\end{center}
%%Adds affiliations from YAML
\begin{justify}
\footnotesize \emph{ 
\\*
\textsuperscript{1}US Environmental Protection Agency, Office of Research and Development,
National Health and Environmental Effects Research Laboratory, Atlantic
Ecology Division, 27 Tarzwell Drive Narragansett, RI, 02882, USA\\*
\\*
\textsuperscript{2}Big Name University, Department of R, City, BN, 01020, USA\\*
\\*
\textsuperscript{3}Estacion Biologica de Donana (EBD-CSIC), E-41092 Sevilla, Spain\\*
}
%%Adds corresponding author email(s) from YAML
\newcounter{num}
\setcounter{num}{1}
\\[0.1cm]
\footnotesize \emph{ 
\ifnum\value{num}=1%
\textsuperscript{*} corresponding author:
\fi
\href{mailto:barragansljose@gmail.com}{\nolinkurl{barragansljose@gmail.com}}
\stepcounter{num}
}
\end{justify}
%%Adds date from YAML
\normalsize

\end{singlespace}


\singlespace

\vspace{2mm}\hrule

Pollinator sharing can have negative consequences for species fitness
with the arrival of foreign pollen. However, the costs of heterospecific
pollen are not yet well understood. For this reason, we have conducted a
glasshouse experiment where we try to understand how phylogenetic
relatedness and the different traits of these species are involved in
this process. We experimentally crossed 10 species belonging to three
different families: Brassicaceae, Solanaceae and Convolvulaceae.
Overall, more than 4000 crosses were done and seed set and pollen tubes
were considered as proxy of effect. We found that for all species
foreign pollen (50\% or less) reduced seed set. Moreover, the seed set
reduction is not dependent on the degree of relatedness of the pollen
donor. However, the effect is governed by the degree of relatedness and
the traits of the species recipient. Our results show that the outcome
of heterospecific pollen deposition is determined in greater degree by
the traits of the pollen recipient than the pollen donor and that
certain traits such as compatibility system are crucial to understand
the costs of heterospecific pollen.

\vspace{3mm}\hrule

\emph{Keywords}: heterospecific pollen, plant reproduction, fitness,
interspecific competition, phylogenetic distance.

\doublespace

\bleft

\section{INTRODUCTION}\label{introduction}

\textbf{Paragraph 1} General idea to our concept

In natural systems plant species normally coexist and share their floral
visitors with other species Bascompte et al. (2003). This pollinator
sharing from the plant perspective at the pre-pollination stage can be
negative due to competition Pauw (2013) or positive due to facilitation
Carvalheiro et al. (2014). Once the floral visitor has arrived to the
flower, pollen deposition on the stigma can take place and hence ovule
fertilization. An increasing number of visits generally correlates with
higher chances of fertilization Engel and Irwin (2003). However this is
not always the case, among these possible flower visitors we find also
nectar robbers and pollen thiefs Inouye (1980) and the quality of pollen
that is deposit on the stigma is also highly relevant to the pollination
succes Aizen and Harder (2007). Moreover, other less study issues in the
pollination process are conspecific pollen loss and the arrival of
foreign pollen which can have important detrimental effects on species
fitness Morales and Traveset (2008) Ashman and Arceo-Gómez (2013).

\textbf{Paragraph 2} Introducing topic and knowledge gap

Recent studies have advanced in the ecological understanding of
heterospecific pollen effect Morales and Traveset (2008) Ashman and
Arceo-Gómez (2013) Arceo-Gómez and Ashman (2016). A general overview of
foreign pollen arrival is that it can play an important role on species
fitness but seems to be context dependent and not always produce a
decrease in fitness Morales and Traveset (2008). Part of this
unpredictability is due to the enormous variability of foreing pollen
transferred in nature, where levels between 0 and 75 percent are seen,
but most commonly values ranges between 10 and 20 percent of the total
pollen load, being the generalist species the ones that receive greater
loads of heterospecific pollen Montgomery and Rathcke (2012) Fang and
Huang (2013). Although heterospecific pollen quantity is fundamental to
understand the outcome of the interaction so is the different traits of
both pollen donor and recipient. Ashman and Arceo-Gómez (2013)
postulated the first predictive framework for traits of heterospecific
pollen effect where traits such as compatibility system and pollen size
among others should be crucial to understand foreing pollen effect.
Moreover, in Tong and Huang (2016) a assymetric effect was shown in a
crossing experiment between 6 species of the genus \emph{Pedicularis}
where the pollen of long styled species was able to grow the ful length
of the style on short styled species but not viceversa. Despite these
recent caveats, we still lack empirical evidence to affirm what are the
main traits that drive heterospecific pollen effect for both pollen
donor and recipient at seed production level. Interestingly, this trait
based question of effect between a pair of species cannot be solved
without consider the phylogenetic relatedness of them. A general view is
that close related species will have greater negative impact than far
related species (REFS) but few studies test this fact. The interaction
between close related species have been tradionally overlooked (Refs)
but not always the effect of congeners or confamilias can be seen due to
these other species are capable of fertilization (Refs.) Combining both
traits and phylogenetic relatedness could give a better picture of how
both parts interact in the heterospecific pollen process.

\textbf{Paragraph 3} Expanding ideas with examples

Examples of articles\ldots{}

can reduce species fitness (REFS) but seems to be highly
contex-dependent. There are hypothesized that some traits can play a
crucial role in this species interaction such as stigma type, pollen
size,

Mention invasive species in this paragraph

Few studies have tried to understand how relatedness is involved in the
hp effect but generally Until our knowledge \ldots{}.

Rescue from here the useful things:

Invasive species are supposed to have greater negative effect than
native ones Arceo-Gómez and Ashman (2016). Although when non-natives
species don´t have greater negative effect we still don´t know why. For
this reason, this ecological question is non a native non native one is
a trait based issue that is still to be solved. Moreover, the quantity
of pollen that integrates in the network can be quite variable ranging
from low quatities Bartomeus et al. (2008) to intermediate (ref) to high
(ref). Moreover, closely related species are supposed to reduce fitness
in greater effect but the evidence is scarce and based on independent
studies with different methodologies (Arceo-gomez \& Ashman 2016) or
studies that just check it with a pair of species that are highly
related with the aim to understand hybridization costs (refs). There is
a need to deepen into how relatednes is involve in the costs of
heterospecific pollen effect. Furthermore, following the conceptual
trait framework of Ashman and Arceo-Gomez on heterospecific pollen there
are good theoretical basis for trait effect. Notwithstanding, non
empirical work has tested how really these traits are involved in
heterospecific pollen effect.

Explain traits. Put examples

what is closely related? same genus? Just that right, the rest is far
related?

I would like to add that the experiments focus on two proxies of effect
prezygotic and postzygotic. Why focus on postzygotic? Is the final stage
where we can see the effect. Further studies should also study
germination rates.

Traditionally heterospecific pollen effect has focused its attention on
different pollen donors as a main driver of different effect. However in
this article we want to emphasize that this is true for the cases that
the species are higly close related where pollen recognition can take
place (eg hybridization) but not when this pollen is from less closely
related species which the main driver of effect is determined by the
reproductive biology of the female part of the plant(compatibility
system, stigma type, stigma area and number of ovules).

\textbf{Paragraph 4} Introducing our experiment

Sell well our work: We are the first empirical experiment testig the
effect of heterospecific pollen with phylogenetic distance

The great difficulty of working with pollen in a coflowering community
make the understanding of heterospecifc pollen effect a real challenge.
For this reason we have created an artificial co-flowering community in
a glasshouse to test the effect with all the possible combinations among
them. Where we test the folowing hypothesis: 1) Does heterospecific
pollen reduce seed set, if so, 2) Does heterospecific pollen effect
depend on the relatedness of the species, 3) Does heterospecific pollen
effect depend on any floral trait?

Maybe another possible hypothesis to test is the reciprocity of the
effect of heterospecifc pollen????

Use the sterile species as a proof of the mechanical interference. Was a
mistake but seems cool proof!!!

\section{METHODS}\label{methods}

\href{}{comment starts} Glasshouse trial • Species selected and why --
how you made them co-flower • Give details of sources and planting
seeds, growth medium in pots, temperature and light details • Hand
crosses and how you did them, how you measured seed set over time. •
Analyses of data -- standardization, means, matrices etc.

• Analyses and technical difficulties: We calculated effect size by
subtracting the mean of the cross pollinated seed set by the mean effect
of the HP pollen (explain exactly what figures you used to calculate
this) -- check with liam about potentially using missing values analyses
for the species we don't have?

Check that the method is working well to prove that your crosses were
close to 50\% results in SI i.e not all mixes were 50/50\% and we have
now counted all the pollen to make this a quantitative variable. We also
need to factor in the point that we have different total abundances of
pollen across our treatments, irrespective of ratios. To what extent are
differences in the ratios of pollen applied by hand across different
plant families influenced by plant traits such as pollen size,
morphology and stigma surface type?

Results -- may need to include amount of pollen in models as random
factor- prefill matrix with missing value analyse for the species you
don't have.\\
Question 1: how do different pollination treatments (100\% HP, 50\% HP,
self and cross ) impact HP pollen across different plant families? Even
with 100\% HP one (or more species?) still produced seed set.

Result Effect size of Seed set \textasciitilde{} phylogenetic distance
relationship We found that the variation ?/ mean effect size of seed set
is positively related to phylogenetic distance. This means the more
unrelated the species are, the greater the negative impact of
heterospecific pollen (give stats effect size i.e.~Procrustes, X = 0.35;
P = 0.03)

Question 2 : what are the main traits impacting HP impacts?
(compatibility system, pollen size, stigma surface, wet/dry stigma,
length of style etc.

Effect size of seed set \textasciitilde{} floral traits/ reproductive
plant traits We found that the three best terms to explain the variation
in seed set is pollen/ovule ratio, stigma width and style length (Stats
effect size i.e.~X = 0.39, P = 0.02).\\
Need to provide correlation matrix for all traits just for 10 species
Show both ways to present this. Which particular traits do you find
significant effects for? Show this and give stats. Present plot for each
trait and effect size

\href{}{comment finishes}

The study was conducted in a glasshouse at University of New England
(Armidale, Australia) from November 2017 to March 2018. Rooms were
temperature controlled depending on the requirements of the species with
day and night temperature differences. The species selected (Table 1)
belonged to three different families, Solanaceae, Brassicaceae and
Convolvulaceae. The criteria of species/family selection was based on
close/distant related species (see phylogenetic tree for relatedness fig
1), heterogeneous traits, low structural flower complexity and fast life
cycle. For the purpose of the experiment all the species where
considered as pollen recipient and as pollen donor (see interaction
matrix, fig 2). Species were watered once or twice per day and
fertilized weekly (NPK 23: 3.95: 14).

Brown and Mitchell 2001 could be a good paper to explain why we pick
seed set as a proxy and not fruit set. We cannot see changes on it,
losing information with it.

\textbf{Hand-pollination}

Foreign pollen effect was studied through two different treatments, one
with 50\% conspecific pollen and 50\% heterospecific pollen and a second
one with 100\% foreign pollen (N=10). Seed set was the proxy of effect
(see Brown and Mitchell 2001, for differences in effect between seed set
and fruit set) and ``pollen tubes''. Moreover, hand cross pollination,
hand self pollination, apomixis (bagged emasculated flowers) and natural
selfing were tested (N=10). Flowers were emasculated the day prior
anthesis and hand pollinated next day with a toothpick. Had-pollination
was realized with 3-4 gentle touches on the surface of the stigma. The
mixes of pollen were performed on an eppendorf based on the pollen
counts maded with Neubaeur chamber (each anther was counted 4 times for
20 different anthers per species).

\textbf{Evolutive distance}

Two types of evolutive distances were calculated with MEGA7 thow kinds
of markers: 1) Internal transcribed spacer (ITS) and 2)
ribulose-bisphosphate carboxylase (RBCL)

\textbf{Traits}

Several traits of the ten species were measured. Pollen per anther was
counted, number of ovules, stigma width and length and stigmatic area,
style width and length, ovary width and length. Moreover stigma type was
tested. Self-incompatibility was

We used the statistical language \texttt{R} (R Core Team 2018) for all
our analyses. These were implemented in dynamic rmarkdown documents
using \texttt{knitr} (Xie 2014, 2015, 2018) and \texttt{rmarkdown}
(Allaire et al. 2018) packages. All the multilevel models were fitted
with \texttt{lme4} (Bates et al. 2015).

\section{RESULTS}\label{results}

\section{DISCUSSION}\label{discussion}

Discussion

\begin{enumerate}
\def\labelenumi{\arabic{enumi}.}
\tightlist
\item
  What are the implications of the findings?
\end{enumerate}

\section{CONCLUSIONS}\label{conclusions}

\section{ACKNOWLEDGEMENTS}\label{acknowledgements}

\section{REFERENCES}\label{references}

\hypertarget{refs}{}
\hypertarget{ref-aizen2007}{}
Aizen, M. A., and L. D. Harder. 2007. Expanding the limits of the
pollen-limitation concept: Effects of pollen quantity and quality.
Ecology 88:271--281.

\hypertarget{ref-Allaire_2018}{}
Allaire, J., Y. Xie, J. McPherson, J. Luraschi, K. Ushey, A. Atkins, H.
Wickham, J. Cheng, and W. Chang. 2018. Rmarkdown: Dynamic documents for
r.

\hypertarget{ref-arceo2016}{}
Arceo-Gómez, G., and T.-L. Ashman. 2016. Invasion status and
phylogenetic relatedness predict cost of heterospecific pollen receipt:
Implications for native biodiversity decline. Journal of Ecology
104:1003--1008.

\hypertarget{ref-ashman2013}{}
Ashman, T.-L., and G. Arceo-Gómez. 2013. Toward a predictive
understanding of the fitness costs of heterospecific pollen receipt and
its importance in co-flowering communities. American Journal of Botany
100:1061--1070.

\hypertarget{ref-bartomeus2008}{}
Bartomeus, I., J. Bosch, and M. Vilà. 2008. High invasive pollen
transfer, yet low deposition on native stigmas in a carpobrotus-invaded
community. Annals of Botany 102:417--424.

\hypertarget{ref-bascompte2003}{}
Bascompte, J., P. Jordano, C. J. Melián, and J. M. Olesen. 2003. The
nested assembly of plant--animal mutualistic networks. Proceedings of
the National Academy of Sciences 100:9383--9387.

\hypertarget{ref-Bates_2015}{}
Bates, D., M. Mächler, B. Bolker, and S. Walker. 2015. Fitting linear
mixed-effects models using lme4. Journal of Statistical Software
67:1--48.

\hypertarget{ref-carvalheiro2014}{}
Carvalheiro, L. G., J. C. Biesmeijer, G. Benadi, J. Fründ, M. Stang, I.
Bartomeus, C. N. Kaiser-Bunbury, M. Baude, S. I. Gomes, V. Merckx, and
others. 2014. The potential for indirect effects between co-flowering
plants via shared pollinators depends on resource abundance,
accessibility and relatedness. Ecology letters 17:1389--1399.

\hypertarget{ref-engel2003}{}
Engel, E. C., and R. E. Irwin. 2003. Linking pollinator visitation rate
and pollen receipt. American Journal of Botany 90:1612--1618.

\hypertarget{ref-fang2013}{}
Fang, Q., and S.-Q. Huang. 2013. A directed network analysis of
heterospecific pollen transfer in a biodiverse community. Ecology
94:1176--1185.

\hypertarget{ref-inouye1980}{}
Inouye, D. W. 1980. The terminology of floral larceny. Ecology
61:1251--1253.

\hypertarget{ref-montgomery2012}{}
Montgomery, B. R., and B. J. Rathcke. 2012. Effects of floral
restrictiveness and stigma size on heterospecific pollen receipt in a
prairie community. Oecologia 168:449--458.

\hypertarget{ref-morales2008}{}
Morales, C. L., and A. Traveset. 2008. Interspecific pollen transfer:
Magnitude, prevalence and consequences for plant fitness. Critical
Reviews in Plant Sciences 27:221--238.

\hypertarget{ref-pauw2013}{}
Pauw, A. 2013. Can pollination niches facilitate plant coexistence?
Trends in ecology \& evolution 28:30--37.

\hypertarget{ref-R_Core_Team_2018}{}
R Core Team. 2018. R: A language and environment for statistical
computing. R Foundation for Statistical Computing, Vienna, Austria.

\hypertarget{ref-tong2016}{}
Tong, Z.-Y., and S.-Q. Huang. 2016. Pre-and post-pollination interaction
between six co-flowering pedicularis species via heterospecific pollen
transfer. New Phytologist 211:1452--1461.

\hypertarget{ref-Xie_2014}{}
Xie, Y. 2014. Knitr: A comprehensive tool for reproducible research in
R. \emph{in} V. Stodden, F. Leisch, and R. D. Peng, editors.
Implementing reproducible computational research. Chapman; Hall/CRC.

\hypertarget{ref-Xie_2015}{}
Xie, Y. 2015. Dynamic documents with R and knitr. 2nd editions. Chapman;
Hall/CRC, Boca Raton, Florida.

\hypertarget{ref-Xie_2018}{}
Xie, Y. 2018. Knitr: A general-purpose package for dynamic report
generation in r.

\eleft

\clearpage

\listoftables

\newpage

\newpage

\clearpage

\listoffigures

\newpage

\newpage

\blandscape

\elandscape

\clearpage

\end{document}