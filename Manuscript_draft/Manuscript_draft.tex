%%%%%%%%%%%%%%%%%%%%%%%%%%%%%%%%%%%%%%%%%%%%%%%%%%%%%%%%%%%%%%%%%%%%%%%%%%%%%
%% Original default rstudio/pandoc latex file
%% upated by @jhollist 09/15/2014
%% inspired by @cboetting https://github.com/cboettig/template and
%% @rmflight blog posts:
%% http://rmflight.github.io/posts/2014/07/analyses_as_packages.html 
%% http://rmflight.github.io/posts/2014/07/vignetteAnalysis.html).  
%%%%%%%%%%%%%%%%%%%%%%%%%%%%%%%%%%%%%%%%%%%%%%%%%%%%%%%%%%%%%%%%%%%%%%%%%%%%%

\documentclass[11pt,a4paper]{article}
\usepackage[T1]{fontenc}
\usepackage{lmodern}
\usepackage{amssymb,amsmath}
\usepackage{ifxetex,ifluatex}
\usepackage{fixltx2e} % provides \textsubscript
% use upquote if available, for straight quotes in verbatim environments
\IfFileExists{upquote.sty}{\usepackage{upquote}}{}
\ifnum 0\ifxetex 1\fi\ifluatex 1\fi=0 % if pdftex
  \usepackage[utf8]{inputenc}
\else % if luatex or xelatex
  \ifxetex
    \usepackage{mathspec}
    \usepackage{xltxtra,xunicode}
  \else
    \usepackage{fontspec}
  \fi
  \defaultfontfeatures{Mapping=tex-text,Scale=MatchLowercase}
  \newcommand{\euro}{€}
\fi
% use microtype if available
\IfFileExists{microtype.sty}{\usepackage{microtype}}{}
\usepackage[margin=1in]{geometry}
\usepackage{longtable,booktabs}
\usepackage{graphicx}
% Redefine \includegraphics so that, unless explicit options are
% given, the image width will not exceed the width of the page.
% Images get their normal width if they fit onto the page, but
% are scaled down if they would overflow the margins.
\makeatletter
\def\ScaleIfNeeded{%
  \ifdim\Gin@nat@width>\linewidth
    \linewidth
  \else
    \Gin@nat@width
  \fi
}
\makeatother
\let\Oldincludegraphics\includegraphics
{%
 \catcode`\@=11\relax%
 \gdef\includegraphics{\@ifnextchar[{\Oldincludegraphics}{\Oldincludegraphics[width=\ScaleIfNeeded]}}%
}%
\ifxetex
  \usepackage[setpagesize=false, % page size defined by xetex
              unicode=false, % unicode breaks when used with xetex
              xetex]{hyperref}
\else
  \usepackage[unicode=true]{hyperref}
\fi
\hypersetup{breaklinks=true,
            bookmarks=true,
            pdfauthor={},
            pdftitle={Recipient plant traits are important determinants of the impacts of heterospecific pollen upon plant reproduction},
            colorlinks=true,
            citecolor=blue,
            urlcolor=blue,
            linkcolor=magenta,
            pdfborder={0 0 0}}
\urlstyle{same}  % don't use monospace font for urls
\setlength{\parindent}{0pt}
\setlength{\parskip}{6pt plus 2pt minus 1pt}
\setlength{\emergencystretch}{3em}  % prevent overfull lines
\setcounter{secnumdepth}{0}

%%%%%%%%%%%%%%%%%%%%%%%%%%%%%%%%%%%%%%%%%%%%%%%%%%%%%%%%
%Changes borrowed from @cboettig, added by @jhollist 
% A modified page layout 
\textwidth 6.75in
\oddsidemargin -0.15in
\evensidemargin -0.15in
\textheight 9in
\topmargin -0.5in
\usepackage{lineno} % add 
  \linenumbers % turns line numbering on 
%%%%%%%%%%%%%%%%%%%%%%%%%%%%%%%%%%%%%%%%%%%%%%%%%%%%%%%%

%%%%%%%%%%%%%%%%%%%%%%%%%%%%%%%%%%%%%%%%%%%%%%%%%%%%%%%%
%%Packages and layout changes by @jhollist 09/15/2014
\usepackage{ragged2e}
\usepackage[font=normalsize]{caption}
  \usepackage[doublespacing]{setspace}
\usepackage{parskip}
\usepackage{fancyhdr}
\pagestyle{fancy}
\fancyhf{}
\renewcommand{\headrulewidth}{0pt}
  \rfoot{\today}
\lfoot{\thepage}
%%Changed default abstract width and added lines
\renewenvironment{abstract}{
  \hfill\begin{minipage}{1\textwidth}
  \rule{\textwidth}{1pt}\vspace{5pt}
  \normalsize
  \begin{justify}
  \bfseries\abstractname\vspace{5pt}
  \end{justify}}
  {\par\noindent\rule{\textwidth}{1pt}\end{minipage}
}
%%%%%%%%%%%%%%%%%%%%%%%%%%%%%%%%%%%%%%%%%%%%%%%%%%%%%%%%

\title{Recipient plant traits are important determinants of the impacts of
heterospecific pollen upon plant reproduction}
\author{
Jose B. Lanuza
Ignasi Bartomeus
Tia-Lynn Ashman
Romina Rader
}
\date{}
% Allowing for landscape pages
\usepackage{lscape}
\newcommand{\blandscape}{\begin{landscape}}
\newcommand{\elandscape}{\end{landscape}}

% Left justification of the text: see https://www.sharelatex.com/learn/Text_alignment
% \usepackage[document]{ragged2e} % already in the latex template
\newcommand{\bleft}{\begin{flushleft}}
\newcommand{\eleft}{\end{flushleft}}

%%Fix tightlist error: https://stackoverflow.com/questions/40438037/tightlist-error-using-pandoc-with-markdown
%%Added 2018-03-26 
\providecommand{\tightlist}{%
  \setlength{\itemsep}{0pt}\setlength{\parskip}{0pt}}
%%%  
  

\begin{document}
%%Edited by @jhollist 09/15/2014
%%Adds title from YAML
\begin{singlespace}
\begin{center}
\huge Recipient plant traits are important determinants of the impacts of
heterospecific pollen upon plant reproduction
\end{center}
%%Adds Author, correspond email asterisk, and affilnum from YAML
\begin{center}
\large
Jose B. Lanuza \textsuperscript{*} \textsuperscript{1} 
Ignasi Bartomeus \textsuperscript{2} 
Tia-Lynn Ashman \textsuperscript{3} 
Romina Rader \textsuperscript{1} 
\end{center}
%%Adds affiliations from YAML
\begin{justify}
\footnotesize \emph{ 
\\*
\textsuperscript{1}University of New England (Australia)\\*
\\*
\textsuperscript{2}Estacion Biologica de Donana (EBD-CSIC), E-41092 Sevilla, Spain\\*
\\*
\textsuperscript{3}Department of Biological Sciences, University of Pittsburgh 4249 Fifth
Avenue, Pittsburgh, Pennsylvania 15260-3929 USA\\*
}
%%Adds corresponding author email(s) from YAML
\newcounter{num}
\setcounter{num}{1}
\\[0.1cm]
\footnotesize \emph{ 
\ifnum\value{num}=1%
\textsuperscript{*} corresponding author:
\fi
\href{mailto:barragansljose@gmail.com}{\nolinkurl{barragansljose@gmail.com}}
\stepcounter{num}
}
\end{justify}
%%Adds date from YAML
\normalsize

\end{singlespace}


\singlespace

\vspace{2mm}\hrule

Pollinator sharing can have negative consequences for plant fitness with
the arrival of foreign pollen. However, the mechanisms underlying the
variation in outcomes as a result of pollen contribution by different
plant species are not yet well understood. We conducted a glasshouse
experiment to understand how plant traits and phylogenetic relatedness
mediate the impacts of 15 heterospecific pollen transfer. We conducted
1800 reciprocal crosses by experimentally transferring pollen (50\% and
100\% foreign pollen ratio) between 10 species belonging to three
different families: Brassicaceae, Solanaceae and Convolvulaceae. Seed
set was used as proxy of plant fitness. In the treatments of 100\%
foreign pollen, we found reduced seed set in X\% of the treatments. In
the treatments of 50\% foreign pollen, we found reduced seed set in 65\%
of the treatments. Moreover, the reduction in seed set was dependent on
the reproductive traits of the pollen recipient, but not the pollen
donor or relatedness. Our results show that certain traits of recipient
plants, particularly compatibility system, are critical in understanding
the costs of heterospecific pollen.

\vspace{3mm}\hrule

\emph{Keywords}: heterospecific pollen, plant reproduction, fitness,
interspecific competition, phylogenetic distance.

\doublespace

\bleft

\section{INTRODUCTION}\label{introduction}

In most ecosystems, plant species normally coexist and share their
floral visitors with other species Waser et al. (1996); Carvalheiro et
al. (2014). From the plants' perspective, pollinator sharing can be
positive for some plants as an increasing number of visits often
correlates with higher chances of fertilization Engel and Irwin (2003).
Yet, among these possible flower visitors there are also nectar robbers,
pollen thieves Inouye (1980); Magrach et al. (2017), and inconstant
pollinators that transfer foreign pollen from other plants Pauw (2013).
By visiting many plant species, many pollinators are responsible for
conspecific pollen loss and the transport of foreign pollen, both of
which can have important detrimental effects on species fitness Morales
and Traveset (2008); Ashman and Arceo-Gómez (2013); Arceo-Gómez and
Ashman (2016). Receiving both sufficient quantity and quality deposited
on the stigma is thus highly relevant to the pollination success of the
plant Aizen and Harder (2007). Foreign pollen arrival can play an
important role in plant species fitness but outcomes are variable and
appear to be context dependent as there is not always a decrease in
fitness Morales and Traveset (2008). Some of this variation is likely
due to the enormous variability of foreign pollen transferred across
systems ranging from 0 to 75 percent. However, most studies report
ranges of heterospecific pollen between 0 and 20 percent of the total
pollen load Bartomeus et al. (2008) Montgomery and Rathcke (2012);
Ashman and Arceo-Gómez (2013); Fang and Huang (2013), yet even these
relatively low amounts of heterospecific pollen transferred can decrease
fitness greatly Thomson et al. (1982).

While we have some understanding of the impacts of heterospecific pollen
quantity, we have little knoweldge of the factors that could be driving
the variation in pollen quality upon fitness. Plant traits are crucial
to understand heterospecific pollen effect but the multifactorial nature
of the traits that are involved in the pollen-pistil interaction make
difficult to unravel exactly which traits are driving the effect. Ashman
and Arceo-Gómez (2013) postulated the first predictive framework that
identifies a need to understand how plant traits might mediate
heterospecific pollen effect, whereby mating system and pollen size were
predicted to potentially mediate the impact of foreign pollen transfer
on plant fitness.

The concept of trait driven mechanisms is not new and is supported by
system specific studies. Pollen size, pollen aperture number and pollen
allelopathy are thought to be key components in understanding the
outcome of foreign pollen arrival Murphy and Aarssen (1995); Ashman and
Arceo-Gómez (2013). For example, small pollen is predicted to decrease
plant fitness because XXXXXX. Yet, large pollen can outcompete smaller
pollen grains due to faster pollen tube growth rate Williams and Rouse
(1990). Hence, understanding the different mechanical or chemical
effects of pollen requires knowledge of the female traits of the pollen
recipient to also be considered Montgomery and Rathcke (2012); Ashman
and Arceo-Gómez (2013); Tong and Huang (2016). For example, greater
stigmatic area is positively correlated with greater heterospecific
pollen deposition Montgomery and Rathcke (2012) and therefore likely to
result in an greater negative effect upon plant fitness. Further,
species that are self-incompatible are thought to be more resistant to
the negative impacts of heterospecific pollen than self-compatible
species Ashman and Arceo-Gómez (2013).

When both donor and recipient traits are considered together, other
combinations of traits are also likely to impact plant fitness. For
example, large pollen grains could potentially clog small stigmas with
fewer pollen grains, and larger stigmas are less likely to be clogged by
small pollen grains. Yet, few studies have considered how effects might
differ among donor and recipient species. Tong and Huang (2016)
demonstrate an asymmetrical effect in 6 species of Pedicularis whereby
foreign pollen of long styled species was able to grow the full length
of the style on short styled species but not vice versa. While this
suggests that the impacts of heterospecific pollen may differ among
pollen donor and recipient, few studies have been conducted to ascertain
whether this pattern is in fact a general trend or to identify the
extent to which other plant traits are critical to heterospecific pollen
impacts.

It is challenging to identify general patterns with respect to the
mechanisms driving foreign pollen impacts as results are often obscured
by the variability within and among species. Closely related species are
more likely to have similar traits Letten and Cornwell (2015). The
similarity in traits between closely related species can lead tohigher
chances of ovule usurpation/abortion Arceo-Gómez and Ashman (2011)
hence, greater negative effects of HP pollen are thought to be a
associated with more closely related species Ashman and Arceo-Gómez
(2013); Arceo-Gómez and Ashman (2016). Few studies however, have focused
on the impacts of heterospecific pollen on fitness of distantly related
species Galen and Gregory (1989); Neiland and Wilcock (1999) and those
that have, often report low sample sizes and a lack of significance.
Therefore, there is a need to study the effect of heterospecific pollen
of far and close related species at community level beyond single
pairwise interactions. Given that pollen carried on many insects and
stigmas has been found to carry multiple species of foreign pollen with
little attention to degree of relatedness Arceo-Gómez and Ashman (2016);
Fang and Huang (2013). understanding the role of foreign pollen from
distantly related species thus deserves greater attention

We investigated how floral reproductive traits and relatedness mediate
the impact of heterospecific pollen by creating an artificial
co-flowering community in a glasshouse with 10 species belonging to
three different families with heterogeneous reproductive traits. Our
study addressed the following questions:

\begin{enumerate}
\def\labelenumi{\arabic{enumi}.}
\item
  To what extent does the amount of foreign pollen applied to stigmas
  impact plant reproductive fitness (i.e.~50\% and 100\% foreign pollen
  ratio.
\item
  How do floral reproductive traits and plant relatedness mediate the
  impacts of heterospecific pollen on seed set.
\end{enumerate}

\section{METHODS}\label{methods}

The study was conducted in a glasshouse at University of New England
(Armidale, Australia) from November 2017 to March 2018. Rooms were
temperature controlled depending on the requirements of the species with
day and night temperature differences. The experimental design had
species from three different families: Brassicaceae, Convolvulaceae and
Solanaceae (\textbf{Table 1}). The species of the study had different
reproductive traits and different degree of relatedness (see
phylogenetic tree, \textbf{Figure 1}) where the reciprocal crosses
between species allowed us to have multiple different scenarios of both
traits and relatedness. Moreover, the species selected had fast life
cycle and low structural flower complexity in order to perform the
pollination treatments and grow the different species from seeds. For
the purpose of the experiment all the species where considered as pollen
recipient and as pollen donor (see interaction matrix, fig 2). Species
were watered once or twice per day and fertilized weekly (NPK 23: 3.95:
14) and the rooms of the glasshouse were temperature controlled with
temperature oscillations between day and night.

\newpage

\begin{figure}
\includegraphics[width=1\linewidth]{images/phylo_image} \caption{Phylogenetic tree of the ten species used in the experiment from three different families from top to bottom: Brassicaceae, Convolvulaceae and Solanaceae}\label{fig:unnamed-chunk-1}
\end{figure}

\newpage

\textbf{Table 1} Species list with family and genus.

\begin{longtable}[]{@{}lll@{}}
\toprule
Family & Genus & Species\tabularnewline
\midrule
\endhead
Brassicaceae & Brassica & Brassica rapa\tabularnewline
Brassicaceae & Brassica & Brassica oleracea\tabularnewline
Brassicaceae & Eruca & Eruca versicaria\tabularnewline
Brassicaceae & Sinapis & Sinapis alba\tabularnewline
Convolvulaceae & Ipomoea & Ipomoea aquatica\tabularnewline
Convolvulaceae & Ipomoea & Ipomoea purpurea\tabularnewline
Solanaceae & Capsicum & Capsicum annuum\tabularnewline
Solanaceae & Petunia & Petunia integrifolia\tabularnewline
Solanaceae & Solanum & Solanum lycopersicum\tabularnewline
Solanaceae & Solanum & Solanum melongena\tabularnewline
\bottomrule
\end{longtable}

\newpage

\textbf{Hand pollination}

Foreign pollen effect was studied through two different treatments, one
with 50\% conspecific pollen and 50\% heterospecific pollen and a second
one with 100\% foreign pollen in order to see if foreign pollen can
trigger fruit production by itself or even seeds through ovule
usurpation. Therefore, we perfomed 180 different heterospecific
treatments (N=10). Seed set was the proxy of effect for all our
treatments. Moreover, hand cross-pollination (between individuals of the
same species), hand self-pollination, apomixis (bagged emasculated
flowers) and natural selfing were tested for each species (N=10). For
the treatments with foreign pollen and hand cross-pollination, flowers
were emasculated the day prior anthesis and hand pollinated next day
with a toothpick. Hand-pollination was conducted with 3-4 gentle touches
on the stigma surface. For each species 20 anthers were collected and
their pollen counted with a hemocytometer, each anther was counted 4
times and then an average of these counts was performed. Once, the
average number of pollen grains per anther was known, the proportion of
anthers per mix was calculated in order to achieve a 50-50\% mix. To
confirm that the treatments applied were the desire proportions, the
total stigmatic load of pollen was counted and the proportions
calculated between the two species of the mix. Because pollen from the
same family was difficult to distinguish, and we expected similar
properties in mixing, we counted pollen from just one randomly selected
species within each donor family different to the focal's family (N=3).

\textbf{Traits and evolutive distance}

The traits measured for each species were pollen per anther, pollen
size, number of ovules and stigma, style, ovary width and length. For
the stigma, the stigmatic area was also measured and moreover the
stigmas were divided in wet/dry type with the help of the
stereomicroscope. All the morphometrical measurements were performed
with a stereophotomicroscope with the exception of pollen size that was
carried out with a light microscope. Pollen was counted for 20 anthers
of each species with 4 replicates per sample with an hemocytometer.
Previously, anthers were squashed on a known solution with the pippete
tip and homogeneize with a vortex for 30 seconds. Ovule number was
counted with the help of a stereomicroscope and a small grid over a
petri dish from 15 randomly selected flowers. Fruits per number of
flowers treated were counted for just Solanaceae species with fleshy
fruit. For all the species we counted the number of seeds produced per
average number of ovules. Levels of self-incompatibility were estimated
by dividing the seed set of hand self-pollination by hand
cross-pollination Lloyd and Schoen (1992).

\textbf{Analysis}

To evaluate heterospecific pollen effect on seed production we performed
linear mixed models. The distinct heterospecific pollination treatments
were compared through relevelling each variable with the cross
pollination treatment which was our control for optimum seed production
for all the species. The different replicates of each treatment were
considered as random effects. Seed production was scaled for all the
species with mean 0 and standard deviation of 1 prior to the analysis.
All the analysis were conducted with the statistical language \texttt{R}
(R Core Team 2018).

To compare the magnitude of effect of heterospecific pollen across
species we conducted standarized Hedges' d {[}(mean of mixed 50\% mix -
mean of cross pollination)/pooled SD{]} with \textbf{effsize} package.
We did in three different ways: effect sizes of each donor per focal
species; effect sizes per family of the different donors per focal
species; effect sizes of all the donors grouped per focal species.

We conducted mantel test to check for correlations between
heterospecific pollen effect and phylogenetic distance. Due to
improvents in statistical power we used the square root of the
phylogenetic distance (Letten \& Cornwell 2014). Two different
phylogenetic distances were used from two kinds of markers: 1) Internal
transcribed spacer (ITS) and 2) ribulose-bisphosphate carboxylase
(RBCL). The sequences were obtained from GenBank
(\url{https://www.ncbi.nlm.nih.gov/genbank/}, accessed 20 Oct. 2018).
The sequences were aligned with clustal omega and the pairwise evolutive
distances calculated with MEGA7.

In order to test the relative effect of traits on seed production with
foreign pollen we performed Mantel test in R (\textbf{vegan} package,
Euclidean distance) between the assymetrical matrix of heterospecific
pollen effect (10 by 10 matrix) with the different distance matrices of
traits. Heterospecific pollen effect was obtained through the
subtraction of seed production by hand cross-pollination minus seed
production of the different heterospecific pollen treatments. To find a
model with the best explanatory traits we used the function
\textbf{bioenv} from R. We also conducted Mantel test between the matrix
of heterospecific pollen effect and the distance matrix from all the
traits. Moreover, we explored also the correlation between traits and
heterospecific pollen effect through generalized mixed models where the
response variable was heterospecific pollen effect and the explanatory
variable the different traits. In addition, we tested the correlation
between the total amount of pollen deposited on the stigma with
heterospecific pollinations and the stigma size through Pearson's
correlation.

\href{Jose}{Phylogenetic signal of traits?}

Total pollen deposited on stigmas was significantly correlated with the
stigmatic area, pearsons correlation=0.57 and p-value=0.008. Proportions
of pollen talk about it to 50-50 mix? Talk aboout this in methods too.

Also the plots of ratios to appendices.

NMDS to appendix?

\section{RESULTS}\label{results}

Results of hand cross-pollination, hand self-pollination, natural
selfing and apomixis are presented in \textbf{Figure 1} (see appendix 1
for table with values). Heterospecific pollen reduced seed set
significatively with the 50-50\% heterospecific pollen treatments for
65\% of the pairwise interactions p\textless{}0.05. Moreover, average
effect sizes differed across species and across families, see
\textbf{Figure 2}. Despite some variability in the effect of the
different pollen donors per species, in general terms the effect of
heterospecific pollen from the distinct nine treatments per species was
homogeneous (see \textbf{Figure 3}), just for four species out of ten,
just one donor did not overlap the confidence intervals with other
donors. Therefore, none of the donors had a clear stronger or weaker
effect across species. When the donors were groped by family not big
differences were seen, just for \emph{S. lycopersicum} the confidence
intervals of Brassicaceae and Convolvulaceae did not overlap (see
\textbf{Figure 4}). In addition, for the 100\% hetrospecific pollen
treatments we did not find almost seed production. However, for just one
species (\emph{S. alba}) the control pollination and the heterospecific
pollination with pollen from a confamilial had similar seed production.
For two Solanaceae species \emph{S. melongena} and \emph{C. annuum}
100\% pollen treatments produced few seedles fruits (3\% and 9\%
respectively) and they did not for the apomictic treatments.

Results of Mantel test between heterospecific pollen effect and
phylogenetic distance gave a positive statistically clear correlation
for both markers (p\textless{}0.05). The correlations with ITS and RBCL
markers was respectively of 0.29 and 0.25. We found a significant
phylogenetic signal of traits for pollen size, stigma measurements and
style length (p\textless{}0.05). Although with a lack of a significant
correlation pagel's lambda values were also relatively high
(\textgreater{}0.45) for incompatibility index, ovary length and levels
of selfing. Moreover, Mantel test between heterospecific pollen effect
and traits gave also a positive significant correlation with a r value
of 0.4. When the effect was look trait by trait with Mantel test, stigma
type and stigma measurements (length, width and area) gave a significant
positive correlation with heterospecific pollen effect.

\href{Jose}{explain ratios and total pollen} \href{}{add table with
morphometrical traits to appendix}

I haven't add this to the draft yet but when I look just Solanaceae
species, interestingly seems to be a negative trend with style length
and heterospecific pollen effect. Curiously, Capsicum and tomato are the
ones that are more affected by hp and both have the shortest stigma and
style. The stigma results of tomato are supported by the plot of effct
sizes per family where convolvulaceae has a greater negative effect than
the other two families (and they are the ones with bigger pollen size)

Here I show the plot of just solanaceae and then with all the species. I
have gouped all the treatments, if not I feel like we are doing p
hacking with all the analysis, because is increasing the strength of
each point 9 times.

\newpage

\begin{figure}

{\centering \includegraphics{output/figures/unnamed-chunk-3-1} 

}

\caption{Barplot with the different treatments that provide information of the reproductive biology of the ten species. The y axis is the proportion of ovules converted to seed in percentage. The different treatments (N=10) which are presented in the legend are, hand cross-pollination, hand self-pollination, natural selfing and apomixis. More information about these treatments can be found in Methods and Appendices.}\label{fig:unnamed-chunk-3}
\end{figure}

\newpage

\begin{figure}
\centering
\includegraphics{output/figures/unnamed-chunk-4-1.pdf}
\caption{The impact of foreign pollen on recipient plant species. Effect
sizes (with 95\%confidence intervals) of 9 different donor species of
heterospecific pollen upon all recipients.}
\end{figure}

\newpage

\begin{figure}
\centering
\includegraphics{output/figures/unnamed-chunk-5-1.pdf}
\caption{The response of heterospecific pollen upon 10 recipient plant
species. Each panel represents one recipient plant species crossed with
50\% mixes of the other 9 species.}
\end{figure}

\newpage

\begin{figure}
\centering
\includegraphics{output/figures/unnamed-chunk-6-1.pdf}
\caption{A}
\end{figure}

\newpage

\begin{figure}
\centering
\includegraphics{output/figures/unnamed-chunk-7-1.pdf}
\caption{Phylogenetic signal of the average heterospecific pollen effect
size per species}
\end{figure}

\newpage

\section{DISCUSSION}\label{discussion}

supporting ideas: Species that are strong selfers or strong outcrossers
have less variablity in mating systems and predictions of effect could
be more realistic (see figure 1 from Whitehead et al. (2018)).

Interestingly the effect of the different donors per species were very
homogeneous which lead us to think that the female reproductive traits
of the pollen recipient are the main ones in explaining heterospecific
pollen effect. Our main predictive trait of effect is stigma size, and
because we found a correlation between the pollen quantity deposited on
the stigma and stigma size, we argue that that the total load of pollen
deposited per treatment can obscured what are the main traits in driving
heterospecific pollen effect.

Curiously between the species with greater effect sizes, we found
completely self incompatible species, species with the smallest stigma
and the species with shortes style. Develop more\ldots{}

Although we have found a positive correlation between phylogenetic
distance and heterospecific pollen effect, this results have to be
treated carefully. From our results we want to highlight that also far
related species can affect negatively fitness but the effect from close
related is species can also have important detrimental effects.
Moreover, the effect of close related species can be masked by the
possibility of hibridization as it occure between two of our species of
Brassica. Moreover, although different effect between distinct donors
can occur we want to note the importance on the traits of the recipient
that determine an homogeneous effect between donors as we have shown in
figure xxx. These different traits will define how it will be the effect
across the different species independently of the nature of the donor
and also differently even between species of . Although traditionally
the nature of the pollen donor have been very studied, the
recipient\ldots{}

Herbs vs tress, annual vs perennial\ldots{} Many flowers vs few flowered
species; structural composition on a system

What are the implications of the findings?

\section{CONCLUSIONS}\label{conclusions}

\section{ACKNOWLEDGEMENTS}\label{acknowledgements}

\section{REFERENCES}\label{references}

\hypertarget{refs}{}
\hypertarget{ref-aizen2007}{}
Aizen, M. A., and L. D. Harder. 2007. Expanding the limits of the
pollen-limitation concept: Effects of pollen quantity and quality.
Ecology 88:271--281.

\hypertarget{ref-arceo2011}{}
Arceo-Gómez, G., and T.-L. Ashman. 2011. Heterospecific pollen
deposition: Does diversity alter the consequences? New Phytologist
192:738--746.

\hypertarget{ref-arceo2016}{}
Arceo-Gómez, G., and T.-L. Ashman. 2016. Invasion status and
phylogenetic relatedness predict cost of heterospecific pollen receipt:
Implications for native biodiversity decline. Journal of Ecology
104:1003--1008.

\hypertarget{ref-ashman2013}{}
Ashman, T.-L., and G. Arceo-Gómez. 2013. Toward a predictive
understanding of the fitness costs of heterospecific pollen receipt and
its importance in co-flowering communities. American Journal of Botany
100:1061--1070.

\hypertarget{ref-bartomeus2008}{}
Bartomeus, I., J. Bosch, and M. Vilà. 2008. High invasive pollen
transfer, yet low deposition on native stigmas in a carpobrotus-invaded
community. Annals of Botany 102:417--424.

\hypertarget{ref-carvalheiro2014}{}
Carvalheiro, L. G., J. C. Biesmeijer, G. Benadi, J. Fründ, M. Stang, I.
Bartomeus, C. N. Kaiser-Bunbury, M. Baude, S. I. Gomes, V. Merckx, and
others. 2014. The potential for indirect effects between co-flowering
plants via shared pollinators depends on resource abundance,
accessibility and relatedness. Ecology letters 17:1389--1399.

\hypertarget{ref-engel2003}{}
Engel, E. C., and R. E. Irwin. 2003. Linking pollinator visitation rate
and pollen receipt. American Journal of Botany 90:1612--1618.

\hypertarget{ref-fang2013}{}
Fang, Q., and S.-Q. Huang. 2013. A directed network analysis of
heterospecific pollen transfer in a biodiverse community. Ecology
94:1176--1185.

\hypertarget{ref-galen1989}{}
Galen, C., and T. Gregory. 1989. Interspecific pollen transfer as a
mechanism of competition: Consequences of foreign pollen contamination
for seed set in the alpine wildflower, polemonium viscosum. Oecologia
81:120--123.

\hypertarget{ref-inouye1980}{}
Inouye, D. W. 1980. The terminology of floral larceny. Ecology
61:1251--1253.

\hypertarget{ref-letten2015}{}
Letten, A. D., and W. K. Cornwell. 2015. Trees, branches and (square)
roots: Why evolutionary relatedness is not linearly related to
functional distance. Methods in Ecology and Evolution 6:439--444.

\hypertarget{ref-lloyd1992}{}
Lloyd, D. G., and D. J. Schoen. 1992. Self-and cross-fertilization in
plants. i. functional dimensions. International Journal of Plant
Sciences 153:358--369.

\hypertarget{ref-magrach2017}{}
Magrach, A., J. P. González-Varo, M. Boiffier, M. Vilà, and I.
Bartomeus. 2017. Honeybee spillover reshuffles pollinator diets and
affects plant reproductive success. Nature ecology \& evolution 1:1299.

\hypertarget{ref-montgomery2012}{}
Montgomery, B. R., and B. J. Rathcke. 2012. Effects of floral
restrictiveness and stigma size on heterospecific pollen receipt in a
prairie community. Oecologia 168:449--458.

\hypertarget{ref-morales2008}{}
Morales, C. L., and A. Traveset. 2008. Interspecific pollen transfer:
Magnitude, prevalence and consequences for plant fitness. Critical
Reviews in Plant Sciences 27:221--238.

\hypertarget{ref-murphy1995}{}
Murphy, S. D., and L. W. Aarssen. 1995. Reduced seed set in elytrigia
repens caused by allelopathic pollen from phleum pratense. Canadian
Journal of Botany 73:1417--1422.

\hypertarget{ref-neiland1999}{}
Neiland, M., and C. Wilcock. 1999. The presence of heterospecific pollen
on stigmas of nectariferous and nectarless orchids and its consequences
for their reproductive success. Protoplasma 208:65--75.

\hypertarget{ref-pauw2013}{}
Pauw, A. 2013. Can pollination niches facilitate plant coexistence?
Trends in ecology \& evolution 28:30--37.

\hypertarget{ref-R_Core_Team_2018}{}
R Core Team. 2018. R: A language and environment for statistical
computing. R Foundation for Statistical Computing, Vienna, Austria.

\hypertarget{ref-thomson1982}{}
Thomson, J. D., B. J. Andrews, and R. Plowright. 1982. The effect of a
foreign pollen on ovule development in diervilla lonicera
(caprifoliaceae). New Phytologist 90:777--783.

\hypertarget{ref-tong2016}{}
Tong, Z.-Y., and S.-Q. Huang. 2016. Pre-and post-pollination interaction
between six co-flowering pedicularis species via heterospecific pollen
transfer. New Phytologist 211:1452--1461.

\hypertarget{ref-waser1996}{}
Waser, N. M., L. Chittka, M. V. Price, N. M. Williams, and J. Ollerton.
1996. Generalization in pollination systems, and why it matters. Ecology
77:1043--1060.

\hypertarget{ref-williams1990}{}
Williams, E., and J. Rouse. 1990. Relationships of pollen size, pistil
length and pollen tube growth rates in rhododendron and their influence
on hybridization. Sexual Plant Reproduction 3:7--17.

\section{APPENDIX}\label{appendix}

\begin{enumerate}
\def\labelenumi{\arabic{enumi}.}
\item
\end{enumerate}

\textbf{Table S1}. Perecentage of seeds produced per ovule for the ten
species used in the experiment. The treatments presented are hand
cross-pollination, hand self-pollination, natural selfing and apomixis
(emasculated flowers).

\begin{longtable}[]{@{}lrrrr@{}}
\toprule
Species & Cross & Self & Natural\_selfing & Apomixis\tabularnewline
\midrule
\endhead
Brassica oleracea & 32.06897 & 0.0000000 & 0.00000 & 0\tabularnewline
Brassica rapa & 44.97041 & 0.0000000 & 0.00000 & 0\tabularnewline
Eruca versicaria & 23.75000 & 0.4166667 & 0.00000 & 0\tabularnewline
Sinapis alba & 43.33333 & 48.3333333 & 5.00000 & 15\tabularnewline
Ipomoea aquatica & 40.00000 & 30.0000000 & 20.00000 & 0\tabularnewline
Ipomoea purpurea & 31.66667 & 86.6666667 & 31.66667 & 0\tabularnewline
Capsicum annuum & 100.00000 & 66.2240664 & 23.48548 & 0\tabularnewline
Petunia integrifolia & 100.00000 & 24.7727273 & 0.00000 &
0\tabularnewline
Solanum lycopersicum & 90.38043 & 43.4782609 & 70.00000 &
0\tabularnewline
Solanum melongena & 60.47525 & 87.9702970 & 21.56436 & 0\tabularnewline
\bottomrule
\end{longtable}

\begin{enumerate}
\def\labelenumi{\arabic{enumi}.}
\setcounter{enumi}{1}
\item
\end{enumerate}

\textbf{Figure S1}. Pollen ratios separated by family A) Brassicaceae,
B) Solanaceae, C) Convolvulaceae. Pollen was counted for crosses with
species of the other two families (N=3). To prepare these ratios all the
pollen grains on the stigma were counted.

\includegraphics{output/figures/unnamed-chunk-9-1.pdf}
\includegraphics{output/figures/unnamed-chunk-9-2.pdf}
\includegraphics{output/figures/unnamed-chunk-9-3.pdf}

\eleft

\clearpage

\listoftables

\newpage

\newpage

\clearpage

\listoffigures

\newpage

\newpage

\blandscape

\elandscape

\clearpage

\end{document}