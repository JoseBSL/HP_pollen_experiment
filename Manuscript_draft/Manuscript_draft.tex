%%%%%%%%%%%%%%%%%%%%%%%%%%%%%%%%%%%%%%%%%%%%%%%%%%%%%%%%%%%%%%%%%%%%%%%%%%%%%
%% Original default rstudio/pandoc latex file
%% upated by @jhollist 09/15/2014
%% inspired by @cboetting https://github.com/cboettig/template and
%% @rmflight blog posts:
%% http://rmflight.github.io/posts/2014/07/analyses_as_packages.html 
%% http://rmflight.github.io/posts/2014/07/vignetteAnalysis.html).  
%%%%%%%%%%%%%%%%%%%%%%%%%%%%%%%%%%%%%%%%%%%%%%%%%%%%%%%%%%%%%%%%%%%%%%%%%%%%%

\documentclass[11pt,a4paper]{article}
\usepackage[T1]{fontenc}
\usepackage{lmodern}
\usepackage{amssymb,amsmath}
\usepackage{ifxetex,ifluatex}
\usepackage{fixltx2e} % provides \textsubscript
% use upquote if available, for straight quotes in verbatim environments
\IfFileExists{upquote.sty}{\usepackage{upquote}}{}
\ifnum 0\ifxetex 1\fi\ifluatex 1\fi=0 % if pdftex
  \usepackage[utf8]{inputenc}
\else % if luatex or xelatex
  \ifxetex
    \usepackage{mathspec}
    \usepackage{xltxtra,xunicode}
  \else
    \usepackage{fontspec}
  \fi
  \defaultfontfeatures{Mapping=tex-text,Scale=MatchLowercase}
  \newcommand{\euro}{€}
\fi
% use microtype if available
\IfFileExists{microtype.sty}{\usepackage{microtype}}{}
\usepackage[margin=1in]{geometry}
\ifxetex
  \usepackage[setpagesize=false, % page size defined by xetex
              unicode=false, % unicode breaks when used with xetex
              xetex]{hyperref}
\else
  \usepackage[unicode=true]{hyperref}
\fi
\hypersetup{breaklinks=true,
            bookmarks=true,
            pdfauthor={},
            pdftitle={How phylogenetic relatedness and floral traits are involved in heterospecifc pollen effect in an artificial co-flowering community},
            colorlinks=true,
            citecolor=blue,
            urlcolor=blue,
            linkcolor=magenta,
            pdfborder={0 0 0}}
\urlstyle{same}  % don't use monospace font for urls
\setlength{\parindent}{0pt}
\setlength{\parskip}{6pt plus 2pt minus 1pt}
\setlength{\emergencystretch}{3em}  % prevent overfull lines
\setcounter{secnumdepth}{0}

%%%%%%%%%%%%%%%%%%%%%%%%%%%%%%%%%%%%%%%%%%%%%%%%%%%%%%%%
%Changes borrowed from @cboettig, added by @jhollist 
% A modified page layout 
\textwidth 6.75in
\oddsidemargin -0.15in
\evensidemargin -0.15in
\textheight 9in
\topmargin -0.5in
\usepackage{lineno} % add 
  \linenumbers % turns line numbering on 
%%%%%%%%%%%%%%%%%%%%%%%%%%%%%%%%%%%%%%%%%%%%%%%%%%%%%%%%

%%%%%%%%%%%%%%%%%%%%%%%%%%%%%%%%%%%%%%%%%%%%%%%%%%%%%%%%
%%Packages and layout changes by @jhollist 09/15/2014
\usepackage{ragged2e}
\usepackage[font=normalsize]{caption}
  \usepackage[doublespacing]{setspace}
\usepackage{parskip}
\usepackage{fancyhdr}
\pagestyle{fancy}
\fancyhf{}
\renewcommand{\headrulewidth}{0pt}
  \rfoot{\today}
\lfoot{\thepage}
%%Changed default abstract width and added lines
\renewenvironment{abstract}{
  \hfill\begin{minipage}{1\textwidth}
  \rule{\textwidth}{1pt}\vspace{5pt}
  \normalsize
  \begin{justify}
  \bfseries\abstractname\vspace{5pt}
  \end{justify}}
  {\par\noindent\rule{\textwidth}{1pt}\end{minipage}
}
%%%%%%%%%%%%%%%%%%%%%%%%%%%%%%%%%%%%%%%%%%%%%%%%%%%%%%%%

\title{How phylogenetic relatedness and floral traits are involved in
heterospecifc pollen effect in an artificial co-flowering community}
\author{
Jose B. Lanuza
true
}
\date{}
% Allowing for landscape pages
\usepackage{lscape}
\newcommand{\blandscape}{\begin{landscape}}
\newcommand{\elandscape}{\end{landscape}}

% Left justification of the text: see https://www.sharelatex.com/learn/Text_alignment
% \usepackage[document]{ragged2e} % already in the latex template
\newcommand{\bleft}{\begin{flushleft}}
\newcommand{\eleft}{\end{flushleft}}

%%Fix tightlist error: https://stackoverflow.com/questions/40438037/tightlist-error-using-pandoc-with-markdown
%%Added 2018-03-26 
\providecommand{\tightlist}{%
  \setlength{\itemsep}{0pt}\setlength{\parskip}{0pt}}
%%%  
  

\begin{document}
%%Edited by @jhollist 09/15/2014
%%Adds title from YAML
\begin{singlespace}
\begin{center}
\huge How phylogenetic relatedness and floral traits are involved in
heterospecifc pollen effect in an artificial co-flowering community
\end{center}
%%Adds Author, correspond email asterisk, and affilnum from YAML
\begin{center}
\large
Jose B. Lanuza \textsuperscript{1,2} 
true
\end{center}
%%Adds affiliations from YAML
\begin{justify}
\footnotesize \emph{ 
\\*
\textsuperscript{1}US Environmental Protection Agency, Office of Research and Development,
National Health and Environmental Effects Research Laboratory, Atlantic
Ecology Division, 27 Tarzwell Drive Narragansett, RI, 02882, USA\\*
\\*
\textsuperscript{2}Big Name University, Department of R, City, BN, 01020, USA\\*
\\*
\textsuperscript{3}Estacion Biologica de Donana (EBD-CSIC), E-41092 Sevilla, Spain\\*
}
%%Adds corresponding author email(s) from YAML
\newcounter{num}
\setcounter{num}{1}
\\[0.1cm]
\footnotesize \emph{ 
\ifnum\value{num}=1%
\textsuperscript{*} corresponding author:
\fi
\href{mailto:barragansljose@gmail.com}{\nolinkurl{barragansljose@gmail.com}}
\stepcounter{num}
}
\end{justify}
%%Adds date from YAML
\normalsize

\end{singlespace}


\singlespace

\vspace{2mm}\hrule

Possible journals to publish: New phytologist, journal of ecology,
oikos\ldots{}

\vspace{3mm}\hrule

\emph{Keywords}: heterospecific pollen, plant reproduction, fitness,
competition

\doublespace

\bleft

\section{INTRODUCTION}\label{introduction}

\textbf{Paragraph 1}

In natural systems plant species normally coexist and share their floral
visitors with other species (Bascompte et al., 2003). This pollinator
sharing from the plant perpective can be negative due to competition
(refs) or positive due to facilitation (refs). Moreover, once the
pollinator has landed on the stigma some other issues for the species
fitness may arise, the arrival of foreign pollen and conspecific pollen
loss (Morales \& Traveset 2008)

\textbf{Paragraph 2}

The effect of heterospecific pollen has been widely studied (Morales \&
Traveset 2008). Invasive species are supposed to have greater negative
effect than native ones Arceo-Gómez and Ashman (2016). Although when
non-natives species don´t have greater negative effect we still don´t
know why. For this reason, this ecological question is non a native non
native one is a trait based issue that is still to be solved. Moreover,
the quantity of pollen that integrates in the network can be quite
variable ranging from low quatities Bartomeus et al. (2008) to
intermediate (ref) to high (ref). Moreover, closely related species are
supposed to reduce fitness in greater effect but the evidence is scarce
and based on independent studies with different methodologies
(Arceo-gomez \& Ashman 2016) or studies that just check it with a pair
of species that are highly related with the aim to understand
hybridization costs (refs). There is a need to deepen into how
relatednes is involve in the costs of heterospecific pollen effect.
Furthermore, following the conceptual trait framework of Ashman and
Arceo-Gomez on heterospecific pollen there are good theoretical basis
for trait effect. Notwithstanding, non empirical work has tested how
really these traits are involved in heterospecific pollen effect.

Explain traits. Put examples

\textbf{Paragraph 3}

\textbf{Paragraph 4}

The great difficulty of working with pollen in a coflowering community
make the understanding of heterospecifc pollen effect a real challenge.
For this reason we have created an artificial co-flowering community in
a glasshouse to test the effect with all the possible combinations among
them. Where we test the folowing hypothesis: 1) Does heterospecific
pollen reduce seed set, if so, 2) Does heterospecific pollen effect
depend on the relatedness of the species, 3) Does heterospecific pollen
effect depend on any floral trait?

\section{METHODS}\label{methods}

The study was conducted in a glasshouse at University of New England
(Armidale, Australia) from November 2017 to March 2018. Rooms were
temperature controlled depending on the requirements of the species with
day and night temperature differences. The species selected (Table 1)
belonged to three different families, Solanaceae, Brassicaceae and
Convolvulaceae. The criteria of species/family selection was based on
close/distant related species (see phylogenetic tree for relatedness fig
1), heterogeneous traits, low structural flower complexity and fast life
cycle. For the purpose of the experiment all the species where
considered as pollen recipient and as pollen donor (see interaction
matrix, fig 2). Species were watered once or twice per day and
fertilized weekly (NPK 23: 3.95: 14).

Brown and Mitchell 2001 could be a good paper to explain why we pick
seed set as a proxy and not fruit set. We cannot see changes on it,
losing information with it.

\textbf{Hand-pollination}

Foreign pollen effect was studied through two different treatments, one
with 50\% conspecific pollen and 50\% heterospecific pollen and a second
one with 100\% foreign pollen (N=10). Seed set was the proxy of effect
(see Brown and Mitchell 2001, for differences in effect between seed set
and fruit set) and ``pollen tubes''. Moreover, hand cross pollination,
hand self pollination, apomixis (bagged emasculated flowers) and natural
selfing were tested (N=10). Flowers were emasculated the day prior
anthesis and hand pollinated next day with a toothpick. Had-pollination
was realized with 3-4 gentle touches on the surface of the stigma. The
mixes of pollen were performed on an eppendorf based on the pollen
counts maded with Neubaeur chamber (each anther was counted 4 times for
20 different anthers per species).

\textbf{Evolutive distance}

Two types of evolutive distances were calculated with MEGA7 thow kinds
of markers: 1) Internal transcribed spacer (ITS) and 2)
ribulose-bisphosphate carboxylase (RBCL)

\textbf{Traits}

Several traits of the ten species were measured. Pollen per anther was
counted, number of ovules, stigma width and length and stigmatic area,
style width and length, ovary width and length. Moreover stigma type was
tested. Self-incompatibility was

We used the statistical language \texttt{R} (R Core Team 2018) for all
our analyses. These were implemented in dynamic rmarkdown documents
using \texttt{knitr} (Xie 2014, 2015, 2018) and \texttt{rmarkdown}
(Allaire et al. 2018) packages. All the multilevel models were fitted
with \texttt{lme4} (Bates et al. 2015).

\section{RESULTS}\label{results}

\section{DISCUSSION}\label{discussion}

Discuss.

\section{CONCLUSIONS}\label{conclusions}

\section{ACKNOWLEDGEMENTS}\label{acknowledgements}

\section{REFERENCES}\label{references}

\hypertarget{refs}{}
\hypertarget{ref-Allaire_2018}{}
Allaire, J., Y. Xie, J. McPherson, J. Luraschi, K. Ushey, A. Atkins, H.
Wickham, J. Cheng, and W. Chang. 2018. Rmarkdown: Dynamic documents for
r.

\hypertarget{ref-arceo2016}{}
Arceo-Gómez, G., and T.-L. Ashman. 2016. Invasion status and
phylogenetic relatedness predict cost of heterospecific pollen receipt:
Implications for native biodiversity decline. Journal of Ecology
104:1003--1008.

\hypertarget{ref-bartomeus2008}{}
Bartomeus, I., J. Bosch, and M. Vilà. 2008. High invasive pollen
transfer, yet low deposition on native stigmas in a carpobrotus-invaded
community. Annals of Botany 102:417--424.

\hypertarget{ref-Bates_2015}{}
Bates, D., M. Mächler, B. Bolker, and S. Walker. 2015. Fitting linear
mixed-effects models using lme4. Journal of Statistical Software
67:1--48.

\hypertarget{ref-R_Core_Team_2018}{}
R Core Team. 2018. R: A language and environment for statistical
computing. R Foundation for Statistical Computing, Vienna, Austria.

\hypertarget{ref-Xie_2014}{}
Xie, Y. 2014. Knitr: A comprehensive tool for reproducible research in
R. \emph{in} V. Stodden, F. Leisch, and R. D. Peng, editors.
Implementing reproducible computational research. Chapman; Hall/CRC.

\hypertarget{ref-Xie_2015}{}
Xie, Y. 2015. Dynamic documents with R and knitr. 2nd editions. Chapman;
Hall/CRC, Boca Raton, Florida.

\hypertarget{ref-Xie_2018}{}
Xie, Y. 2018. Knitr: A general-purpose package for dynamic report
generation in r.

\eleft

\clearpage

\listoftables

\newpage

\newpage

\clearpage

\listoffigures

\newpage

\newpage

\blandscape

\elandscape

\clearpage

\end{document}