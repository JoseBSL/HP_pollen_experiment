%%%%%%%%%%%%%%%%%%%%%%%%%%%%%%%%%%%%%%%%%%%%%%%%%%%%%%%%%%%%%%%%%%%%%%%%%%%%%
%% Original default rstudio/pandoc latex file
%% upated by @jhollist 09/15/2014
%% inspired by @cboetting https://github.com/cboettig/template and
%% @rmflight blog posts:
%% http://rmflight.github.io/posts/2014/07/analyses_as_packages.html 
%% http://rmflight.github.io/posts/2014/07/vignetteAnalysis.html).  
%%%%%%%%%%%%%%%%%%%%%%%%%%%%%%%%%%%%%%%%%%%%%%%%%%%%%%%%%%%%%%%%%%%%%%%%%%%%%

\documentclass[11pt,a4paper]{article}
\usepackage[T1]{fontenc}
\usepackage{lmodern}
\usepackage{amssymb,amsmath}
\usepackage{ifxetex,ifluatex}
\usepackage{fixltx2e} % provides \textsubscript
% use upquote if available, for straight quotes in verbatim environments
\IfFileExists{upquote.sty}{\usepackage{upquote}}{}
\ifnum 0\ifxetex 1\fi\ifluatex 1\fi=0 % if pdftex
  \usepackage[utf8]{inputenc}
\else % if luatex or xelatex
  \ifxetex
    \usepackage{mathspec}
    \usepackage{xltxtra,xunicode}
  \else
    \usepackage{fontspec}
  \fi
  \defaultfontfeatures{Mapping=tex-text,Scale=MatchLowercase}
  \newcommand{\euro}{€}
\fi
% use microtype if available
\IfFileExists{microtype.sty}{\usepackage{microtype}}{}
\usepackage[margin=1in]{geometry}
\ifxetex
  \usepackage[setpagesize=false, % page size defined by xetex
              unicode=false, % unicode breaks when used with xetex
              xetex]{hyperref}
\else
  \usepackage[unicode=true]{hyperref}
\fi
\hypersetup{breaklinks=true,
            bookmarks=true,
            pdfauthor={},
            pdftitle={working title Compatibility system and stygma size of pollen recipient as main predictors of heterospecific pollen effect},
            colorlinks=true,
            citecolor=blue,
            urlcolor=blue,
            linkcolor=magenta,
            pdfborder={0 0 0}}
\urlstyle{same}  % don't use monospace font for urls
\setlength{\parindent}{0pt}
\setlength{\parskip}{6pt plus 2pt minus 1pt}
\setlength{\emergencystretch}{3em}  % prevent overfull lines
\setcounter{secnumdepth}{0}

%%%%%%%%%%%%%%%%%%%%%%%%%%%%%%%%%%%%%%%%%%%%%%%%%%%%%%%%
%Changes borrowed from @cboettig, added by @jhollist 
% A modified page layout 
\textwidth 6.75in
\oddsidemargin -0.15in
\evensidemargin -0.15in
\textheight 9in
\topmargin -0.5in
\usepackage{lineno} % add 
  \linenumbers % turns line numbering on 
%%%%%%%%%%%%%%%%%%%%%%%%%%%%%%%%%%%%%%%%%%%%%%%%%%%%%%%%

%%%%%%%%%%%%%%%%%%%%%%%%%%%%%%%%%%%%%%%%%%%%%%%%%%%%%%%%
%%Packages and layout changes by @jhollist 09/15/2014
\usepackage{ragged2e}
\usepackage[font=normalsize]{caption}
  \usepackage[doublespacing]{setspace}
\usepackage{parskip}
\usepackage{fancyhdr}
\pagestyle{fancy}
\fancyhf{}
\renewcommand{\headrulewidth}{0pt}
  \rfoot{\today}
\lfoot{\thepage}
%%Changed default abstract width and added lines
\renewenvironment{abstract}{
  \hfill\begin{minipage}{1\textwidth}
  \rule{\textwidth}{1pt}\vspace{5pt}
  \normalsize
  \begin{justify}
  \bfseries\abstractname\vspace{5pt}
  \end{justify}}
  {\par\noindent\rule{\textwidth}{1pt}\end{minipage}
}
%%%%%%%%%%%%%%%%%%%%%%%%%%%%%%%%%%%%%%%%%%%%%%%%%%%%%%%%

\title{working title Compatibility system and stygma size of pollen recipient
as main predictors of heterospecific pollen effect}
\author{
Jose B. Lanuza, Ignasi Bartomeus, Tia-Lynn Ashman, Romina Rader
}
\date{}
% Allowing for landscape pages
\usepackage{lscape}
\newcommand{\blandscape}{\begin{landscape}}
\newcommand{\elandscape}{\end{landscape}}

% Left justification of the text: see https://www.sharelatex.com/learn/Text_alignment
% \usepackage[document]{ragged2e} % already in the latex template
\newcommand{\bleft}{\begin{flushleft}}
\newcommand{\eleft}{\end{flushleft}}

%%Fix tightlist error: https://stackoverflow.com/questions/40438037/tightlist-error-using-pandoc-with-markdown
%%Added 2018-03-26 
\providecommand{\tightlist}{%
  \setlength{\itemsep}{0pt}\setlength{\parskip}{0pt}}
%%%  
  

\begin{document}
%%Edited by @jhollist 09/15/2014
%%Adds title from YAML
\begin{singlespace}
\begin{center}
\huge working title Compatibility system and stygma size of pollen recipient
as main predictors of heterospecific pollen effect
\end{center}
%%Adds Author, correspond email asterisk, and affilnum from YAML
\begin{center}
\large
Jose B. Lanuza, Ignasi Bartomeus, Tia-Lynn Ashman, Romina Rader \textsuperscript{*} \textsuperscript{1,2,3} 
\end{center}
%%Adds affiliations from YAML
\begin{justify}
\footnotesize \emph{ 
\\*
\textsuperscript{1}US Environmental Protection Agency, Office of Research and Development,
National Health and Environmental Effects Research Laboratory, Atlantic
Ecology Division, 27 Tarzwell Drive Narragansett, RI, 02882, USA\\*
\\*
\textsuperscript{2}Big Name University, Department of R, City, BN, 01020, USA\\*
\\*
\textsuperscript{3}Estacion Biologica de Donana (EBD-CSIC), E-41092 Sevilla, Spain\\*
}
%%Adds corresponding author email(s) from YAML
\newcounter{num}
\setcounter{num}{1}
\\[0.1cm]
\footnotesize \emph{ 
\ifnum\value{num}=1%
\textsuperscript{*} corresponding author:
\fi
\href{mailto:barragansljose@gmail.com}{\nolinkurl{barragansljose@gmail.com}}
\stepcounter{num}
}
\end{justify}
%%Adds date from YAML
\normalsize

\end{singlespace}


\singlespace

\vspace{2mm}\hrule

Pollinator sharing can have negative consequences for species fitness
with the arrival of foreign pollen. However, the costs of heterospecific
pollen are not yet well understood. For this reason, we have conducted a
glasshouse experiment where we try to understand how phylogenetic
relatedness and the different traits of these species are involved in
this process. We experimentally crossed 10 species belonging to three
different families: Brassicaceae, Solanaceae and Convolvulaceae.
Overall, more than 4000 crosses were done and seed set and pollen tubes
were considered as proxy of effect. We found that for all species
foreign pollen (50\% or less) reduced seed set. Moreover, the seed set
reduction is not dependent on the degree of relatedness of the pollen
donor. However, the effect is governed by the degree of relatedness and
the traits of the species recipient. Our results show that the outcome
of heterospecific pollen deposition is determined in greater degree by
the traits of the pollen recipient than the pollen donor and that
certain traits such as compatibility system are crucial to understand
the costs of heterospecific pollen.

\vspace{3mm}\hrule

\emph{Keywords}: heterospecific pollen, plant reproduction, fitness,
interspecific competition, phylogenetic distance.

\doublespace

\bleft

\section{INTRODUCTION}\label{introduction}

\textbf{Paragraph 1} General idea to our concept

In natural systems plant species normally coexist and share their floral
visitors with other species Bascompte et al. (2003). This pollinator
sharing from the plant perspective at the pre-pollination stage can be
negative due to competition Pauw (2013) or positive due to facilitation
Carvalheiro et al. (2014). Once the floral visitor has arrived to the
flower, pollen deposition on the stigma can take place and hence ovule
fertilization. An increasing number of visits generally correlates with
higher chances of fertilization Engel and Irwin (2003). However this is
not always the case, among these possible flower visitors we find also
nectar robbers and pollen thiefs Inouye (1980) and the quality of pollen
that is deposit on the stigma is also highly relevant to the pollination
succes Aizen and Harder (2007). Moreover, other less study issues in the
pollination process are conspecific pollen loss and the arrival of
foreign pollen which can have important detrimental effects on species
fitness Morales and Traveset (2008) Ashman and Arceo-Gómez (2013).

\textbf{Paragraph 2} Introducing topic and knowledge gap

Recent studies have advanced in the ecological understanding of
heterospecific pollen effect Morales and Traveset (2008) Ashman and
Arceo-Gómez (2013) Arceo-Gómez and Ashman (2016). A general overview of
foreign pollen arrival is that it can play an important role on species
fitness but seems to be context dependent and not always produce a
decrease in fitness Morales and Traveset (2008). Part of this
unpredictability is due to the enormous variability of foreing pollen
transferred in nature, where levels between 0 and 75 percent are seen,
but most commonly values ranges between 0 and 20 percent of the total
pollen load, being the generalist species the ones that receive greater
loads of heterospecific pollen Bartomeus et al. (2008) Montgomery and
Rathcke (2012) Ashman and Arceo-Gómez (2013) Fang and Huang (2013).
Although heterospecific pollen quantity is fundamental to understand the
outcome of the interaction so is the different traits of both pollen
donor and recipient. Ashman and Arceo-Gómez (2013) postulated the first
predictive framework for traits of heterospecific pollen effect, where
different traits such as compatibility system and pollen size among
others seems to be crucial to understand foreing pollen effect.
Moreover, in Tong and Huang (2016) an assymetric effect was shown in a
crossing experiment between 6 species of the genus \emph{Pedicularis}
where the pollen of long styled species was able to grow the full length
of the style on short styled species but not viceversa. Despite these
recent caveats, we still lack empirical evidence to affirm what are the
main traits that drive heterospecific pollen effect for both pollen
donor and recipient at seed production level. Interestingly, to
comprehend how these traits interact is also crucial to look at the
phylogenetic relatedness of the species. There is a considerable amount
of literature of crosses between close related species Brown and
Mitchell (2001) Arceo-Gómez et al. (2016) Tong and Huang (2016) but few
works focused on heterospecific pollen of far related species. Although
the effect of close related species is predicted to be greater Ashman
and Arceo-Gómez (2013) the presence of pollen of non related species on
multiple species Arceo-Gómez and Ashman (2016) and the higher chances to
coexist with a species that has less niche overlap (Ref) make foreign
pollen from far related species also an important subject of study in
order to understand the importance of heterospecific pollen in natural
systems. Notwithstanding, the effect of heterospecific pollen of far and
close related species at community level remains to be explored.

\textbf{Paragraph 3} Expanding ideas with examples

Interestingly, incompatibility system seems to play an important role in
foreign pollen effect where species that are self incompatible would
have stronger barriers towards heterospecific pollen than self
compatible species. The type of incompatibility, sporophytic or
gametophytic is related with the place of pollen recognition where the
former take place at the sitgma level and the latter occurs within the
style, this last late acting pollen recognition mechanism is associated
with greater negative effect (REF). Remarkably, there is a great
variability in mating systems across populations Whitehead et al. (2018)
and therefore predict an effect of foreign pollen is a bit obscured by
the variability within species, however species that are strong selfers
or strong outcrossers have less variablity in mating systems and
predictions of effect could be more realistic (see figure 1 from
Whitehead et al. (2018)).

\textbf{Paragraph 4} Introducing our experiment

The great environmental variability in natural systems and complexity of
floral structures make heterospecific pollination studies a daunting
task. Moreover, variation in sampling effort have been shown to be
determinant to characterize pollen transfer interactions Arceo-Gómez et
al. (2018). Although plant-pollinator network and pollen network studies
can give a first picture of the importance of foreign pollen is
necessary to address how its effect is shaped with both traits and
relatedness of the species. For this reason, in this study we have
created an artificial co-flowering community with 10 species belonging
to three different families where we try to test the following
questions: 1) Does heterospecific pollen reduce seed set, if so, 2) Does
heterospecific pollen effect depend on the relatedness of the species,
3) Does heterospecific pollen effect depend on any floral trait?

\section{METHODS}\label{methods}

The study was conducted in a glasshouse at University of New England
(Armidale, Australia) from November 2017 to March 2018. Rooms were
temperature controlled depending on the requirements of the species with
day and night temperature differences. The species selected (Table 1)
belonged to three different families, Solanaceae, Brassicaceae and
Convolvulaceae. The criteria of species/family selection was based on
close/distant related species (see phylogenetic tree for relatedness fig
1), heterogeneous traits, low structural flower complexity and fast life
cycle. For the purpose of the experiment all the species where
considered as pollen recipient and as pollen donor (see interaction
matrix, fig 2). Species were watered once or twice per day and
fertilized weekly (NPK 23: 3.95: 14).

Brown and Mitchell 2001 could be a good paper to explain why we pick
seed set as a proxy and not fruit set. We cannot see changes on it,
losing information with it.

\textbf{Hand-pollination}

Foreign pollen effect was studied through two different treatments, one
with 50\% conspecific pollen and 50\% heterospecific pollen and a second
one with 100\% foreign pollen (N=10). Seed set was the proxy of effect
(see Brown and Mitchell 2001, for differences in effect between seed set
and fruit set) and ``pollen tubes''. Moreover, hand cross pollination,
hand self pollination, apomixis (bagged emasculated flowers) and natural
selfing were tested (N=10). Flowers were emasculated the day prior
anthesis and hand pollinated next day with a toothpick. Had-pollination
was realized with 3-4 gentle touches on the surface of the stigma. The
mixes of pollen were performed on an eppendorf based on the pollen
counts maded with Neubaeur chamber (each anther was counted 4 times for
20 different anthers per species).

\textbf{Evolutive distance}

Two types of evolutive distances were calculated with MEGA7 for two
kinds of markers: 1) Internal transcribed spacer (ITS) and 2)
ribulose-bisphosphate carboxylase (RBCL)

\textbf{Traits}

The traits measured for each species were pollen per anther, number of
ovules, stigma width and length and stigmatic area, style width and
length, ovary width and length. Moreover stigma type was tested. Pollen
was counted for 20 anthers of each species with 4 replicates per sample
with an hemocytometer. Previously anthers were squashed on a known
solution with the pippete tip and homogeneize with a vortex for 30
seconds. Ovule number was counted with the help of an stereomicroscope
and a small grid over a petri dish from 15 randomly selected flowers.
The different morphometrical traits were measured with XXXX. Levels of
self incompatibility were estimated by dividing the the fruit set of
hand self pollination by hand cross pollination

We used the statistical language \texttt{R} (R Core Team 2018) for all
our analyses. These were implemented in dynamic rmarkdown documents
using \texttt{knitr} (Xie 2014, 2015, 2018) and \texttt{rmarkdown}
(Allaire et al. 2018) packages. All the multilevel models were fitted
with \texttt{lme4} (Bates et al. 2015).

\section{RESULTS}\label{results}

\section{DISCUSSION}\label{discussion}

Discussion

\begin{enumerate}
\def\labelenumi{\arabic{enumi}.}
\tightlist
\item
  What are the implications of the findings?
\end{enumerate}

\section{CONCLUSIONS}\label{conclusions}

\section{ACKNOWLEDGEMENTS}\label{acknowledgements}

\section{REFERENCES}\label{references}

\hypertarget{refs}{}
\hypertarget{ref-aizen2007}{}
Aizen, M. A., and L. D. Harder. 2007. Expanding the limits of the
pollen-limitation concept: Effects of pollen quantity and quality.
Ecology 88:271--281.

\hypertarget{ref-Allaire_2018}{}
Allaire, J., Y. Xie, J. McPherson, J. Luraschi, K. Ushey, A. Atkins, H.
Wickham, J. Cheng, and W. Chang. 2018. Rmarkdown: Dynamic documents for
r.

\hypertarget{ref-arceo2018}{}
Arceo-Gómez, G., C. Alonso, T.-L. Ashman, and V. Parra-Tabla. 2018.
Variation in sampling effort affects the observed richness of
plant--plant interactions via heterospecific pollen transfer:
Implications for interpretation of pollen transfer networks. American
journal of botany 105:1601--1608.

\hypertarget{ref-arceo2016}{}
Arceo-Gómez, G., and T.-L. Ashman. 2016. Invasion status and
phylogenetic relatedness predict cost of heterospecific pollen receipt:
Implications for native biodiversity decline. Journal of Ecology
104:1003--1008.

\hypertarget{ref-arceo2016can}{}
Arceo-Gómez, G., R. A. Raguso, and M. A. Geber. 2016. Can plants evolve
tolerance mechanisms to heterospecific pollen effects? An experimental
test of the adaptive potential in clarkia species. Oikos 125:718--725.

\hypertarget{ref-ashman2013}{}
Ashman, T.-L., and G. Arceo-Gómez. 2013. Toward a predictive
understanding of the fitness costs of heterospecific pollen receipt and
its importance in co-flowering communities. American Journal of Botany
100:1061--1070.

\hypertarget{ref-bartomeus2008}{}
Bartomeus, I., J. Bosch, and M. Vilà. 2008. High invasive pollen
transfer, yet low deposition on native stigmas in a carpobrotus-invaded
community. Annals of Botany 102:417--424.

\hypertarget{ref-bascompte2003}{}
Bascompte, J., P. Jordano, C. J. Melián, and J. M. Olesen. 2003. The
nested assembly of plant--animal mutualistic networks. Proceedings of
the National Academy of Sciences 100:9383--9387.

\hypertarget{ref-Bates_2015}{}
Bates, D., M. Mächler, B. Bolker, and S. Walker. 2015. Fitting linear
mixed-effects models using lme4. Journal of Statistical Software
67:1--48.

\hypertarget{ref-brown2001}{}
Brown, B. J., and R. J. Mitchell. 2001. Competition for pollination:
Effects of pollen of an invasive plant on seed set of a native congener.
Oecologia 129:43--49.

\hypertarget{ref-carvalheiro2014}{}
Carvalheiro, L. G., J. C. Biesmeijer, G. Benadi, J. Fründ, M. Stang, I.
Bartomeus, C. N. Kaiser-Bunbury, M. Baude, S. I. Gomes, V. Merckx, and
others. 2014. The potential for indirect effects between co-flowering
plants via shared pollinators depends on resource abundance,
accessibility and relatedness. Ecology letters 17:1389--1399.

\hypertarget{ref-engel2003}{}
Engel, E. C., and R. E. Irwin. 2003. Linking pollinator visitation rate
and pollen receipt. American Journal of Botany 90:1612--1618.

\hypertarget{ref-fang2013}{}
Fang, Q., and S.-Q. Huang. 2013. A directed network analysis of
heterospecific pollen transfer in a biodiverse community. Ecology
94:1176--1185.

\hypertarget{ref-inouye1980}{}
Inouye, D. W. 1980. The terminology of floral larceny. Ecology
61:1251--1253.

\hypertarget{ref-montgomery2012}{}
Montgomery, B. R., and B. J. Rathcke. 2012. Effects of floral
restrictiveness and stigma size on heterospecific pollen receipt in a
prairie community. Oecologia 168:449--458.

\hypertarget{ref-morales2008}{}
Morales, C. L., and A. Traveset. 2008. Interspecific pollen transfer:
Magnitude, prevalence and consequences for plant fitness. Critical
Reviews in Plant Sciences 27:221--238.

\hypertarget{ref-pauw2013}{}
Pauw, A. 2013. Can pollination niches facilitate plant coexistence?
Trends in ecology \& evolution 28:30--37.

\hypertarget{ref-R_Core_Team_2018}{}
R Core Team. 2018. R: A language and environment for statistical
computing. R Foundation for Statistical Computing, Vienna, Austria.

\hypertarget{ref-tong2016}{}
Tong, Z.-Y., and S.-Q. Huang. 2016. Pre-and post-pollination interaction
between six co-flowering pedicularis species via heterospecific pollen
transfer. New Phytologist 211:1452--1461.

\hypertarget{ref-whitehead2018}{}
Whitehead, M. R., R. Lanfear, R. J. Mitchell, and J. D. Karron. 2018.
Plant mating systems often vary widely among populations. Frontiers in
Ecology and Evolution 6:38.

\hypertarget{ref-Xie_2014}{}
Xie, Y. 2014. Knitr: A comprehensive tool for reproducible research in
R. \emph{in} V. Stodden, F. Leisch, and R. D. Peng, editors.
Implementing reproducible computational research. Chapman; Hall/CRC.

\hypertarget{ref-Xie_2015}{}
Xie, Y. 2015. Dynamic documents with R and knitr. 2nd editions. Chapman;
Hall/CRC, Boca Raton, Florida.

\hypertarget{ref-Xie_2018}{}
Xie, Y. 2018. Knitr: A general-purpose package for dynamic report
generation in r.

\eleft

\clearpage

\listoftables

\newpage

\newpage

\clearpage

\listoffigures

\newpage

\newpage

\blandscape

\elandscape

\clearpage

\end{document}