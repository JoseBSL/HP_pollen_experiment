\documentclass[]{article}
\usepackage{lmodern}
\usepackage{amssymb,amsmath}
\usepackage{ifxetex,ifluatex}
\usepackage{fixltx2e} % provides \textsubscript
\ifnum 0\ifxetex 1\fi\ifluatex 1\fi=0 % if pdftex
  \usepackage[T1]{fontenc}
  \usepackage[utf8]{inputenc}
\else % if luatex or xelatex
  \ifxetex
    \usepackage{mathspec}
  \else
    \usepackage{fontspec}
  \fi
  \defaultfontfeatures{Ligatures=TeX,Scale=MatchLowercase}
\fi
% use upquote if available, for straight quotes in verbatim environments
\IfFileExists{upquote.sty}{\usepackage{upquote}}{}
% use microtype if available
\IfFileExists{microtype.sty}{%
\usepackage[]{microtype}
\UseMicrotypeSet[protrusion]{basicmath} % disable protrusion for tt fonts
}{}
\PassOptionsToPackage{hyphens}{url} % url is loaded by hyperref
\usepackage[unicode=true]{hyperref}
\hypersetup{
            pdftitle={Supporting Information:},
            pdfborder={0 0 0},
            breaklinks=true}
\urlstyle{same}  % don't use monospace font for urls
\usepackage[margin=1in]{geometry}
\usepackage{graphicx,grffile}
\makeatletter
\def\maxwidth{\ifdim\Gin@nat@width>\linewidth\linewidth\else\Gin@nat@width\fi}
\def\maxheight{\ifdim\Gin@nat@height>\textheight\textheight\else\Gin@nat@height\fi}
\makeatother
% Scale images if necessary, so that they will not overflow the page
% margins by default, and it is still possible to overwrite the defaults
% using explicit options in \includegraphics[width, height, ...]{}
\setkeys{Gin}{width=\maxwidth,height=\maxheight,keepaspectratio}
\IfFileExists{parskip.sty}{%
\usepackage{parskip}
}{% else
\setlength{\parindent}{0pt}
\setlength{\parskip}{6pt plus 2pt minus 1pt}
}
\setlength{\emergencystretch}{3em}  % prevent overfull lines
\providecommand{\tightlist}{%
  \setlength{\itemsep}{0pt}\setlength{\parskip}{0pt}}
\setcounter{secnumdepth}{0}
% Redefines (sub)paragraphs to behave more like sections
\ifx\paragraph\undefined\else
\let\oldparagraph\paragraph
\renewcommand{\paragraph}[1]{\oldparagraph{#1}\mbox{}}
\fi
\ifx\subparagraph\undefined\else
\let\oldsubparagraph\subparagraph
\renewcommand{\subparagraph}[1]{\oldsubparagraph{#1}\mbox{}}
\fi

% set default figure placement to htbp
\makeatletter
\def\fps@figure{htbp}
\makeatother


\title{\textbf{Supporting Information:}}
\author{}
\date{\vspace{-2.5em}}

\begin{document}
\maketitle

\section{Recipient and donor characteristics govern hierarchical
structure in a heterospecific pollen competition network of co-flowering
plants}\label{recipient-and-donor-characteristics-govern-hierarchical-structure-in-a-heterospecific-pollen-competition-network-of-co-flowering-plants}

\textbf{Authors:} Jose B. Lanuza, Ignasi Bartomeus, Tia Lynn Ashman,
Romina Rader

The following Supporting Information is available for this article:

\textbf{Table S1.} Species names, common names, varieties and sources of
the different seeds.

\textbf{Table S2.} Numerical values of all the traits measured for each
species.

\textbf{Table S3.} Seed set in percentage for hand cross-pollination,
hand self-pollination, natural selfing and apomixis for all species.

\textbf{Table S4.} Species x species matrix with the significance of
effect ``yes'' or ``no'' of the different donors on the seed set of the
different recipient species.

\textbf{Table S5.} Estimates, standard error, t-value and P-value of the
effect of the different 9 donors on each recipient species.

\textbf{Table S6.} Number of seeds produced with 100\% foreign pollen
treatments for the different recipient species.

\textbf{Table S7.} Phylogenetic signal and significance for all the
different traits

\textbf{Table S8.} Procrustes analysis results.

\textbf{Figure S1.} Correlation matrix for all the different traits.

\textbf{Figure S2.} Total amount of pollen found for the different
treatments.

\textbf{Figure S3.} Pollen ratios for the different recipient species.

\textbf{Figure S4.} Pollen ratios for the different recipient species by
family.

\textbf{Figure S5.} Statistical comparison of pollen ratios by family as
pollen donor and recipient.

\textbf{Figure S6.} Violin plot of the reproductive biology of the
species.

Figure S7. Grouped effect sizes by family for each recipient species.

\end{document}
